%\subsection{In String Languages}

The operations of insertion and deletion are fundamental in formal language theory,
and generative mechanisms based on them have been considered (with linguistic motivation)
 since a long time ago, see ~\cite{marcus} . 

In general form, an insertion operation means adding a substring to a given string
in a specified context, while a deletion operation means removing a substring of a
given string from a specified context. A finite set of insertion-deletion rules, together
with a set of axioms provide a language generating device (an \emph{Insdel system}): starting 
from the set of initial strings and iterating insertion-deletion operations as defined
by the given rules we obtain a language. The number of axioms, the length of the
inserted or deleted strings, as well as the length of the contexts where these operations
take place are natural descriptional complexity measures in this framework. As expected, 
insertion and deletion operations with context dependence are very powerful,
leading to characterizations of recursively enumerable languages. 


The paper ~\cite{margentern} contains an unexpected result: context-free insertion-deletion systems
with one axiom are already universal, they can generate any recursively enumerable
language. Moreover, this result can be obtained by inserting and deleting strings of
a rather small length, at most three. The same paper stated an open question about
context-free insertion-deletion systems having rules dealing with strings of length at
most two.

In ~\cite{verlan} the author answer this open question and showed that if the length of the
inserted and deleted string is at most two, then such systems generate a particular
subset of the family of context-free languages. They also showed that the traditional
complexity measures for insertion-deletion systems, in particular the size of contexts,
need a revision and we propose new measures based on the total weight.

The insertion and deletion operations with context dependendence are given in ~\cite{paun}. Here the authors 
explained how a string is inserted/deleted between two given strings using the concept of $DNA$ computing.

The $insertion$ process can be described as follows.

Suppose if we add a single stranded $DNA$ sequence of the form $5'-x_1uvx_2z-3'$ to a test tube which contains another single stranded $DNA$ 
sequence of the form $3'-\bar{u}\bar{y}\bar{v}-5'$ where $x_1,x_2,u,v,z$ are strings, $\bar{u},\bar{v}$ are the Wartson-Crick complements 
of the strings $u,v$ and $\bar{y}$ is the complement of some new string $y$. Then the $DNA$ strand will \emph{anneal}, $\bar{u}$ will
stick to $u$ and $\bar{v}$ to $v$, folding $\bar{y}$ as shown in figure~\ref{dna_insert}. If we cut the double stranded subsequence $uv$ by using a 
restricyion enzyme: adding $\bar{z}$, which acts as \emph{primer}, we will get a complete double stranded sequence. We can now seperate the two strands 
by melting the solution and hence we obtain two strands $x_1uyvx_2z$ and $\bar{x_1}\bar{u}\bar{y}\bar{v}\bar{x_2}\bar{z}$. Hence the string $y$ has been
inserted between strings $u$ and $v$.

Similarly by using a mismatching annealing, deletion operation which is controlled by a context, can also perform theoretically. The different steps in deletion process is shown in figure ~\ref{dna_delet}.
\begin{definition}
An Insdel system ~\cite{verlan} is a construct $\gamma = (V, T, A, I, D)$, where $V$ is an alphabet, $T \subseteq V$ ,
$A$ is a finite language over $V$ , and $I$, $D$ are finite sets of triples of the form $(u, \alpha, v)$, of
strings over $V$ . The elements of $T$ are terminal symbols (in contrast, those of $V − T$
are called nonterminals), those of $A$ are axioms, the triples in $I$ are insertion rules,
and those from $D$ are deletion rules. An insertion rule $(u, \alpha, v) \in I$ indicates that
the string $\alpha$ can be inserted in between $u$ and $v$, while a deletion rule $(u, \alpha, v) \in D$
indicates that $\alpha$ can be removed from the context $(u, v)$. Stated otherwise, $(u, \alpha, v) \in I$
corresponds to the rewriting rule $uv \rightarrow u\alpha v$, and $(u, \alpha, v) \in D$ corresponds to the
rewriting rule $u\alpha v \rightarrow uv$. We denote by $\Rightarrow_{ins}$ the relation defined by an insertion
rule (formally, $x \Rightarrow_{ins} y$ iff $x = x_1 uvx_2$ , $y = x_1 u\alpha vx_2$ , for some $(u, \alpha, v) \in I$ and
$x_1 , x_2 \in V^* $) and by 
$\Rightarrow_{del}$ the relation defined by a deletion rule (formally, $x \Rightarrow_{del} y$
iff $x = x_1 u\alpha vx_2$ , $y = x_1 uvx_2$ , for some $(u, \alpha, v) \in D$ and $x_1 , x_2 \in V^*$). We refer by $\Rightarrow$
to any of the relations $\Rightarrow_{ins}$ , $\Rightarrow_{del}$ , and denote by $\Rightarrow^*$ the reflexive and transitive
closure of $\Rightarrow$ (as usual, $\Rightarrow^*$ is the transitive closure of $\Rightarrow$).

The language generated by $\gamma$ is defined by $L(\gamma) = \{w \in T^* | x \Rightarrow^* w, \mbox{ for some } x \in A\}$.

An Insdel system $\gamma = (V, T, A, I, D)$ is said to be of weight $(n, m; p, q)$ if

$n = max\{|\alpha| | (u, \alpha, v) \in I\},\\
m = max\{|u| | (u, \alpha, v) ∈ I or (v, \alpha, u) \in I\},\\
p = max\{|\alpha| | (u, \alpha, v) \in D\},\\
q = max\{|u| | (u, \alpha, v) \in D or (v, \alpha, u) \in D\},\\$
The total weight of $\gamma $ is the sum $m + n + p + q$.
\end{definition}
\begin{example}
Consider the Insdel system $ID = (T, T, A, I, \phi)$, with
$T = \{0, 1\}$, $A = \{01\}$ and $I = \{(0, 01, 1)\}$. Here at each derivation
step the string $01$ is inserted in the middle of the word obtained on the previous step
because this is the only place where the context $01$ occurs. Since we start from $01$ it
is clear that $L(ID) = \{0^n1^n | n \geq 1\}$.
Here the weight of the system is (2, 1 ; 0, 0) and has the total weight equal to 3.
\end{example}

We denote by $INS_n^mDEL_p^q$ , for $n, m, p, q \geq 0$, the family of languages $L(\gamma)$
generated by Insdel systems of weight $(n , m ; p , q )$ such that $n \leq n$, $m \leq m$,
$p \leq p$, $q \leq q$. If some of the parameters $n, m, p, q$ is not specified, then we write
instead the symbol $∗$. Thus, $INS_*^0DEL_*^0$ denotes the family of languages generated
by context-free Insdel systems, i.e., with insertion rules of the form $(\epsilon, \alpha, \epsilon) \in I$ and
deletion rules of the form $(\epsilon, \alpha, \epsilon) \in D$, where $\epsilon$ denotes the empty string. 

\subsection{Comparison with grammar systems}
We denote by $CF$ $INSDEL_{m,p}$ the family of context-free Insdel systems
having the length of the inserted string at most $m$ and the length of the deleted string
at most $p$. For a system $ID\in CF$ $INSDEL_{m,p}$ we shall also say that $ID$ is of size
$(m, p)$.

Context-free Insdel systems have an interesting particularity: insertions and deletions 
are uncontrolled and may happen at any time at any place in a string. This fact
can give an impression that such systems cannot be controlled in order to perform
computations. However, this affirmation is not true. In ~\cite{margentern} it is shown that, in spite of
the above remark, such systems are able to simulate an arbitrary Chomsky grammar.

\begin{example}
Consider a context-free Insdel system $ID_2 = (V, T, \{S\}, I, D)$
of size (3, 2) with $T = \{a, b\}$, $V = T \cup \{S, S'\}$, $I = \{S'aSb, S'ab\}$ and $D = SS'$.
The computation in this system goes as follows. Given a word $w_1 Sw_2$ (initially
$w_1 , w_2 = \epsilon$) the string $S'aSb$ is inserted after $S$, which produces $w_1SS'aSbw_2$ . After
that, the deletion rule eliminates $SS$ and word $w_1aSbw_2$ is obtained. It is easy
to see that this corresponds to rewriting rule $S \rightarrow aSb$. If the string $S'aSb$ is not
inserted immediately after $S$, then symbol $S$ cannot be eliminated, hence it will not
be possible to generate a terminal string.

Therefore, system $ID_2$ simulates the grammar with productions $\{S \rightarrow aSb, S \rightarrow
ab\}$. Hence, $L(ID_2 ) = \{a^nb^n | n \geq 1\}$.
\end{example}
Insdel systems of a “sufficiently large” weight can characterize $RE$, the family
of recursively enumerable languages. 

The following results about the comparison of insdel system recursively enumerable languagesare given in~\cite{paun} with proofs.
\begin{itemize}
\item $RE=INS_3^2 DEL_3^0$
\item $RE=INS_1^2 DEL_1^1$
\item $RE=INS_1^1 DEL_2^0$
\item $RE=INS_*^1 DEL_0^0$
\item $INS_2^2 DEL_0^0$ contains non semilinear languages.
\end{itemize}

In ~\cite{madhu} the author gave the following results

from ~\cite{paun,margentern,verlan,madhu}, we can summerize the results as in table~\ref{ins_tble1}.
\begin{table}
\begin{tabular}{llllll}
No.&  Total weight& (n, m; p, q)& Family generated &References\\
\cline{1-5}
1 &6 &(3, 0; 3, 0)& RE &~\cite{margentern}\\
2 &5 &(1, 2; 1, 1)& RE &~\cite{lila,paun}\\
3 &5 &(1, 2; 2, 0)& RE &~\cite{lila,paun}\\
4 &5 &(2, 1; 2, 0) &RE &~\cite{lila,paun}\\
5 &5 &(1, 1; 1, 2)& RE &~\cite{akihiro}\\
6 &5 &(2, 1; 1, 1)& RE &~\cite{akihiro}\\
7 &5 &(2, 0; 3, 0)& RE &~\cite{margentern}\\
8 &5 &(3, 0; 2, 0) &RE &~\cite{margentern}\\
9 &4 &(1, 1; 2, 0) &RE &~\cite{paun}\\
10 &4 &(1, 1; 1, 1)& RE &~\cite{akihiro}\\
11 &4 &(2, 0; 2, 0)& $\subset$ CF &~\cite{verlan}\\
12 &m+1 &(m, 0; 1, 0)&$\subset$ CF &~\cite{verlan}\\
13 &p+1 &(1, 0; p, 0)&$\subset$ REG &~\cite{verlan}\\
14 &4 &(1, 2; 1, 0)& ? &\\
 

\end{tabular}
\label{ins_tble1}
\caption{Comparison of insdel system and recursively enumerable languages}
\end{table}


%\subsection{In Tree Languages}
\subsection{Contextual Grammars}

\subsection{Tree Adjoining Grammars(TAG)}

\subsection{Contextual Tree Adjoining Grammars(CTAG)}

