

\subsection{Introduction}
   
    Similarly a tree, say $t$ can be inserted as a sub-tree of another tree, say  $T$ based on some context.
    The insertion may be \emph{fixed arity} or \emph{variable arity}. Here arity refers to the arity of nodes in tree $T$.
In this paper we consider the insertion of trees into trees and call it as \emph{tree insertion system}.
%\subsection{Variable Arity Insertion}
%Here the arity of a node in tree $T$ will change after the insertion process. The sub-tree $t$ is attached as a new child of that node in $T$.

%\subsection{Definition}
\begin{definition}
The \emph{blocked tree insertion system} is a tuple
 $\Gamma = (\Sigma,\mathcal{A},\mathcal{A} ',R)$ where,\\
\begin{itemize}
    \item $\Sigma$ is a finite set of \emph{ranked alphabets}.
    \item $\mathcal{A} = \{A_1\cup A_2\cup \cdots \cup A_m\}$, where each $A_i, 1\leq i\leq m$ is a finite set of axioms.\\
%$\{A_1^{[x_1,y_1,z_1]},A_2^{[x_2,y_2,z_2]}\cdots , A_m^{[x_m,y_m,z_m]}\}$: 
%    Finite set of \emph{axioms (trees)}.\\
    {%\small 
With each set $A_i$ is associated  a \emph{\textbf{flag}} $F_i$ which is a triple $[x_i,y_i,z_i]$, 
	where $x_i,y_i,z_i\in\{-1,0,1,\cdots ,k\}$, for some fixed $k$, are integers,  
	which plays some role in language generation 
    unless $x_i=y_i=z_i=-1$. (For each insertion from $A_i$, the $x_i$ value gets incremented if $x_i\leq y_i$ and 
    the $x_i$ value gets decremented if $x_i>y_i$. The $x_i$ will be set to $z_i$ if one insertion from $A_i$ 
    happens when $x_i=y_i$. The tree insertion system is said to be \emph{\textbf{stable}}, if $x_i=y_i$ for all flags with 
    $x_i\leq y_i$ initially and $x_i\neq y_i$ for all flags with $x_i>y_i$ initially.)}
    \item $\mathcal{A} '\subseteq \mathcal{A}$ is a finite set of \emph{initial axioms}.
    \item $R=\{r_1,r_2,\cdots ,r_n\}$ is a finite set of \emph{insertion rules}\\\\
    {%\small
    Each $r_i$, for $1\leq i\leq n$ is of the form $(\chi, C_1,C_2,\cdots ,C_p)$ where, 
    \begin{itemize}    
         \item $\chi = (root,left,right)$ which represents a context.

        \begin{itemize}   
 
            \item $root$  is any node in the tree
            \item $left $ is $i^{th}$ child of $root$.% (- checks for the absence of $left$).% $1\leq i< arity(rt)$
            \item $right$ is  $(i+1)^{th}$ child of $root$.   $0\le i \le arity(root)$ and $p\leq arity(root)$\\
		($-$ checks for the absence of a child). %$0\leq i< arity(rt)$
                  \end{itemize}
        \item $C_i = (X,rt',k),1\leq i\leq p,X\in \mathcal{A}$ 

             \begin{itemize}    
%                \item $A_j$  : $j^{th}$ axiom. $1\leq j\leq m$
                \item $rt'$ is the root of the tree to be attached.%$rt'\notin F$
                \item $k$ is the position at which $rt'$ is to get attached. $1\le k \leq arity(root)$ and it is between the nodes $left$ and $right$.
             \end{itemize}
          \end{itemize}
          {\small \emph{As examples,  $\chi = (a,b,c)$ denotes a node with label $a$ having a node with label $b$ as $i^{th}$ 
         child and a node with label $c$ as $(i+1)^{th}$ child for $1\le i< arity(a)$. \\
   %         $\chi = (a,b,-)$ or $(a,-,b)$ checks for a node $a$ with $b$ as $i^{th}$ child for $1\le i\le arity(a)$, 
    %    without checcking $(i+1)^{th}$ or $(i-1)^{th}$ child. \\
%             $\chi = (a,b,-)$  checks for a node $a$ with $b$ as $i^{th}$ child,$1\le i\leq arity(a)$, 
 %       without checcking $(i+1)^{th}$ child \\
         $\chi = (a,-,-)$ denotes a leaf node with label $a$.}} \\ 
    }
\end{itemize}
\end{definition}
The \textbf{derivation step} is described as follows.

If $r=(\chi ,C_1,C_2,\dots ,C_p)$ is a rule with $\chi =(a,b,c)$ and $C_i =(X,d,k)$ where $X$ is an axiom with trees having
root with label $d$ and $t$ is a tree with root $p$(domain) having label $a$. Then $t\Rightarrow t'$ by rule $r$ where $t'(p.k)$ 
is the tree with root label $d$, $t'(p.i)$ is tree with root label $b$ and $t'(p.(i+p+1))$ is tree with root label $c$.

$\overset{*}\Rightarrow $ is the reflexive transitive closure of $\Rightarrow$.\\

The flag associated with $X$ is also updated when $r$ is applied. For each insertion from $X$, the $x_i$ value gets incremented if $x_i\leq y_i$ ,  
    the $x_i$ value gets decremented if $x_i>y_i$ and the $x_i$ will be set to $z_i$ if $x_i=y_i$.
We describe the derivation informally with an example.

Suppose 
{\small    
\pstree[nodesep=1pt,levelsep=3ex]{\Tr{a}}
    {
            \Tr{a}
	    \Tr{c}	
    }}
is a tree 
{\small
$A_1=
\begin{Bmatrix}
        \pstree[nodesep=1pt,levelsep=3ex]{\Tr{a}}
        {
            \Tr{a}
            \Tr{c}
        }
        ,
        \pstree[nodesep=1pt,levelsep=3ex]{\Tr{b}}
        {
            \Tr{c}
            \Tr{c}
        }
\end{Bmatrix}
$ }be the axiom and \hbox{$r_1=((a,-,-),(A_1,b,1),(A_1,a,2))$} be the insertion rule. Then by using $r_1$  
{\small
  \pstree[nodesep=1pt,levelsep=3ex]{\Tr{a}}
    {
            \Tr{a}
	    \Tr{c}	
    }
$\Rightarrow$
   \pstree[nodesep=1pt,levelsep=3ex]{\Tr{a}}
   {
   \pstree[nodesep=1pt,levelsep=3ex]{\Tr{a}}
    {
    	\pstree[nodesep=1pt,levelsep=3ex]{\Tr{b}}
        {
            \Tr{c}
            \Tr{c}
        }
    	\pstree[nodesep=1pt,levelsep=3ex]{\Tr{a}}
	{
		\Tr{a}
	    	\Tr{c}	
    	}
}
	\Tr{c}
}
}
\begin{comment}
$\Rightarrow$
   \pstree[nodesep=1pt,levelsep=3ex]{\Tr{a}}
   {
   \pstree[nodesep=1pt,levelsep=3ex]{\Tr{a}}
    {
    	\pstree[nodesep=1pt,levelsep=3ex]{\Tr{b}}
        {
            \Tr{c}
            \Tr{c}
        }
    	\pstree[nodesep=1pt,levelsep=3ex]{\Tr{a}}
	{
		\pstree[nodesep=1pt,levelsep=3ex]{\Tr{a}}
		{	
    		\pstree[nodesep=1pt,levelsep=3ex]{\Tr{b}}
    		    {
            		\Tr{c}
            		\Tr{c}
			}
		\pstree[nodesep=1pt,levelsep=3ex]{\Tr{b}}
    		    {
            		\Tr{c}
            		\Tr{c}
			}

        	}

	    	\Tr{c}	
    	}
}
	\Tr{c}
}	
\end{comment}

Here, at leaf node $a$, a subtree with root $b$ is attached as the first child and another subtree with root $a$ is attached as the second child.

The \textbf{language generated} by a  tree insertion system $\Gamma$, represented by $L(\Gamma)$, is the set of trees,  
with each node having children exactly equal to its arity, derivable in $\Gamma$, when it is in \emph{stable stae},
 from an initial axiom, using rules of $\Gamma$.
%all nodes has children exactly as its arity, derivable by $\Gamma$ from any of its initial axiom.
\\\\
$L(\Gamma)=\begin{Bmatrix} t|S\overset{*}\Rightarrow t, S \mbox{ in some } A_i\in \mathcal{A} '. \mbox{ Each node of }t 
\mbox{ has children exactly as its arity}\\ \mbox{ and } \Gamma \mbox{ is in stable state}\end{Bmatrix}$\\

\begin{example}
%\textbf{Example 1:} $L_{r_1}=\{a(b^i(g),c^j(g)),i,j\ge 1\}$\\\\
$L_{r_1}=\{a(b^i(h),c^j(h)),i,j\ge 1\}$\\\\       
$\Gamma _{r_1}= (\Sigma ,\mathcal{A},\mathcal{A} ',R)$ where,\\
$\Sigma =\{a_2,b_1,c_1,h_0\}$, 
$\mathcal{A}=\{A_1,A_2\}$\\$F_1=\{-1,-1,-1\},F_2=\{-1,-1,-1\}$
$\mathcal{A} '=\{A_1\}$, where \\
$A_1=
\begin{Bmatrix}
        \pstree[nodesep=1pt,levelsep=3ex]{\Tr{a}}
        {
            \Tr{b}
            \Tr{c}
        }
        ,
        \pstree[nodesep=1pt,levelsep=3ex]{\Tr{a}}
        {
            \pstree[nodesep=1pt,levelsep=3ex]{\Tr{b}}
            {
                \Tr{h}
            }
            \pstree[nodesep=1pt,levelsep=3ex]{\Tr{c}}
            {
                \Tr{h}
            }
        }

\end{Bmatrix}
,
\hspace{3mm}
A_2=
\begin{Bmatrix}
        \pstree[nodesep=1pt,levelsep=6ex]{\Tr{b}}
        {}
        \hspace{2mm}
        ,
        \hspace{2mm}
        \pstree[nodesep=1pt,levelsep=6ex]{\Tr{c}}
        {}
        \hspace{2mm}
        ,
        \hspace{2mm}
        \pstree[nodesep=1pt,levelsep=6ex]{\Tr{h}}
        {}

\end{Bmatrix}$
\\\\
$R=\{r_1,r_2,r_3,r_4\}$, where\\\\
   $r_1=(\chi _1, C_1)$,  where $\chi _1=(b,-)$ and $C_1= (A_2,b,1)$ \\
   $r_2=(\chi _2, C_2)$,  where $\chi _2=(c,-)$ and $C_2= (A_2,c,1)$ \\
   $r_3=(\chi _1, C_3)$   where $C_3=(A_2,h,1)$\\
   $r_4=(\chi _2, C_3)$\\  
\noindent \rule{\textwidth}{1pt}
\end{example} 

\begin{example}
%\textbf{Example 2:} $L_{r_2}=\{a(b^i(g),c^j(g)),i,j\ge 1\mbox{, } i\%2 =0\}$\\\\
$L_{r_2}=\{a(b^i(h),c^j(h)),i,j\ge 1\mbox{, } i\%2 =0\}$\\\\
$\Gamma _{r_2} = (\Sigma,\mathcal{A},\mathcal{A} ',R)$ where,\\
$\Sigma =\{a_2,b_1,c_1,h_0\}$,  
$\mathcal{A} = \{A_1,A_2\}$,$F_1=[-1,-1,-1],F_2=[-1,-1,-1]$,
$\mathcal{A}'= \{A_1\}$, where\\
{\small
$A_1=
\begin{Bmatrix}
        \pstree[nodesep=1pt,levelsep=3ex]{\Tr{a}}
        {
            \pstree[nodesep=1pt,levelsep=3ex]{\Tr{b}}
            {
                \Tr{b}
            }
            \Tr{c}
        }
        ,
        \pstree[nodesep=1pt,levelsep=3ex]{\Tr{a}}
        {
            \pstree[nodesep=1pt,levelsep=3ex]{\Tr{b}}
            {
                \pstree[nodesep=1pt,levelsep=3ex]{\Tr{b}}
                {
                  \Tr{h}
                }
            }
            \pstree[nodesep=1pt,levelsep=3ex]{\Tr{c}}
            {
                \Tr{h}
            }
        }
 \end{Bmatrix}
 ,
\hspace{3mm}
A_2=
\begin{Bmatrix}
        \pstree[nodesep=1pt,levelsep=4ex]{\Tr{b}}
        {
            \Tr{b}
        }
         \hspace{2mm}
        ,
        \hspace{2mm}
        \pstree[nodesep=1pt,levelsep=6ex]{\Tr{c}}
        {}
        \hspace{2mm}
        ,
        \hspace{2mm}
        \pstree[nodesep=1pt,levelsep=6ex]{\Tr{h}}
        {}
\end{Bmatrix}$}
\\
$R=\{r_1,r_2,r_3,r_4\}$, where\\
   $r_1=(\chi _1, C_1)$, $r_2=(\chi _2, C_2)$, $r_3=(\chi _1, C_3)$, $r_4=(\chi _2, C_3)$, where \\
	$\chi _1=(b,-)$,  $\chi _2=(c,-)$ , $C_1= (A_2,b,1)$ , $C_2= (A_2,c,1)$ , $C_3=(A_2,h,1)$\\
    \noindent \rule{\textwidth}{1pt}
\end{example} 
   

  The \emph{tree insertion} system $\Gamma _{r_2}$, with some minor changes in its axioms can be used to generate 
   the following regular tree languages.\\
   \begin{itemize}
        \item $L_{r_3}=\{a(b^i(h),c^j(h)),i,j\ge 1\mbox{, } j\%2 =0\}$
        \item $L_{r_4}=\{a(b^i(h),c^j(h)),i,j\ge 1\mbox{, } i\%2 \neq 0\}$
        \item $L_{r_5}=\{a(b^i(h),c^j(h)),i,j\ge 1\mbox{, } j\%2 \neq 0\}$
        \item $L_{r_6}=\{a(b^i(h),c^j(h)),i,j\ge 1\mbox{, } i\%2 = 0\mbox{ and }j\%2 =0\}$
        \item $L_{r_7}=\{a(b^i(h),c^j(h)),i,j\ge 1\mbox{, } i\%2 = 0\mbox{ and }j\%2 \neq 0\}$
        \item $L_{r_8}=\{a(b^i(h),c^j(h)),i,j\ge 1\mbox{, } i\%2 \neq 0\mbox{ and }j\%2 = 0\}$
        \item $L_{r_9}=\{a(b^i(h),c^j(h)),i,j\ge 1\mbox{, } i\%2 \neq 0\mbox{ and }j\%2 \neq 0\}$
   \end{itemize}

\begin{example}
 $L_{r_5}=\{a(b^i(h),c^j(h)),i,j\ge 1\mbox{, } i\%3=0,j\%2 =0\}$\\\\
$\Gamma _{r_5} = (F,\mathcal{A},\mathcal{A} ',R)$ where,\\
$F=\{a_2,b_1,c_1,h_0\}$,  
$\mathcal{A} = \{A_1^{[-1,-1,-1]},A_2^{[-1,-1,-1]}\}$,
$\mathcal{A}'= \{A_1\}$, where\\
$A_1=
\begin{Bmatrix}
        \pstree[nodesep=1pt,levelsep=3ex]{\Tr{a}}
        {
            \pstree[nodesep=1pt,levelsep=3ex]{\Tr{b}}
            {
                \pstree[nodesep=1pt,levelsep=3ex]{\Tr{b}}
                {
                    \Tr{b}
                }
            }
            \pstree[nodesep=1pt,levelsep=3ex]{\Tr{c}}
            {
                \Tr{c}
            }
        }
        ,
        \pstree[nodesep=1pt,levelsep=3ex]{\Tr{a}}
        {
            \pstree[nodesep=1pt,levelsep=3ex]{\Tr{b}}
            {
                \pstree[nodesep=1pt,levelsep=3ex]{\Tr{b}}
                {
                    \pstree[nodesep=1pt,levelsep=3ex]{\Tr{b}}
                    {
                        \Tr{h}
                    }
                }
            }
            \pstree[nodesep=1pt,levelsep=3ex]{\Tr{c}}
            {
                \pstree[nodesep=1pt,levelsep=3ex]{\Tr{c}}
                {
                    \Tr{h}
                }
            }
        }
 \end{Bmatrix}
 ,
\hspace{3mm}
A_2=
\begin{Bmatrix}
        \pstree[nodesep=1pt,levelsep=4ex]{\Tr{b}}
        {
            \pstree[nodesep=1pt,levelsep=4ex]{\Tr{b}}
            {
                \Tr{b}
            }
        }
         \hspace{2mm}
        ,
        \hspace{2mm}
        \pstree[nodesep=1pt,levelsep=6ex]{\Tr{c}}
        {
            \Tr{c}
        }
        \hspace{2mm}
        ,
        \hspace{2mm}
        \pstree[nodesep=1pt,levelsep=6ex]{\Tr{h}}
        {}
\end{Bmatrix}$
\\\\
$R=\{r_1,r_2,r_3,r_4\}$, where\\\\
   $r_1=(\chi _1, C_1)$,  where $\chi _1=(b,-)$ and $C_1= (A_2,b,1)$ \\
   $r_2=(\chi _2, C_2)$,  where $\chi _2=(c,-)$ and $C_2= (A_2,c,1)$ \\
   $r_3=(\chi _1, C_3)$   where $C_3=(A_2,h,1)$\\
   $r_4=(\chi _2, C_3)$\\  

\noindent \rule{\textwidth}{1pt}
\end{example}    
\begin{example}
%\textbf{Example 3:} $L_{r_{10}}=\{t|t\mbox{ has  }a\mbox{ as root}\}$\\\\
                     $L_{r_{3}}=\{t|t\mbox{ has  }a\mbox{ as root}\}$\\\\
$\Gamma _{r_{3}} = (\Sigma,\mathcal{A},\mathcal{A} ',R)$ where,\\
$\Sigma = \{a_2,b_2,c_0\}$,  
$\mathcal{A} = \{A_1,A_2\}$, $F_1=[-1,-1,-1],F_2=[-1,-1,-1]$,
$\mathcal{A}'= \{A_1\}$, where\\
{\small $A_1=
    \begin{Bmatrix}
        \pstree[nodesep=1pt,levelsep=4ex]{\Tr{a}}
        {
            \Tr{c}
            \Tr{c}
        }
        ,
        \pstree[nodesep=1pt,levelsep=4ex]{\Tr{a}}
        {
            \Tr{b}
            \Tr{b}
        }
        , 
        \pstree[nodesep=1pt,levelsep=4ex]{\Tr{a}}
        {
            \Tr{b}
            \Tr{c}
        }
        ,       
        \pstree[nodesep=1pt,levelsep=4ex]{\Tr{a}}
        {
            \Tr{c}
            \Tr{b}
        }
        ,
        \pstree[nodesep=1pt,levelsep=4ex]{\Tr{a}}
        {
            \Tr{a}
            \Tr{a}
        }
        ,
        \pstree[nodesep=1pt,levelsep=4ex]{\Tr{a}}
        {
            \Tr{a}
            \Tr{b}
        }
        ,
        \pstree[nodesep=1pt,levelsep=4ex]{\Tr{a}}
        {
            \Tr{a}
            \Tr{c}
        }
        ,
        \pstree[nodesep=1pt,levelsep=4ex]{\Tr{a}}
        {
            \Tr{b}
            \Tr{a}
        }
        ,       
        \pstree[nodesep=1pt,levelsep=4ex]{\Tr{a}}
        {
            \Tr{c}
            \Tr{a}
        }
        \end{Bmatrix}$\\
$A_2=
\begin{Bmatrix}
       \pstree[nodesep=1pt,levelsep=4ex]{\Tr{b}}
        {
            \Tr{c}
            \Tr{c}
        }
        ,
        \pstree[nodesep=1pt,levelsep=4ex]{\Tr{b}}
        {
            \Tr{b}
            \Tr{b}
        }
        , 
        \pstree[nodesep=1pt,levelsep=4ex]{\Tr{b}}
        {
            \Tr{b}
            \Tr{c}
        }
        ,       
        \pstree[nodesep=1pt,levelsep=4ex]{\Tr{b}}
        {
            \Tr{c}
            \Tr{b}
        }
        ,
        \pstree[nodesep=1pt,levelsep=4ex]{\Tr{b}}
        {
            \Tr{a}
            \Tr{a}
        }
        ,
        \pstree[nodesep=1pt,levelsep=4ex]{\Tr{b}}
        {
            \Tr{a}
            \Tr{b}
        }
        ,
        \pstree[nodesep=1pt,levelsep=4ex]{\Tr{b}}
        {
            \Tr{a}
            \Tr{c}
        }
        ,
        \pstree[nodesep=1pt,levelsep=4ex]{\Tr{b}}
        {
            \Tr{b}
            \Tr{a}
        }
        ,       
        \pstree[nodesep=1pt,levelsep=4ex]{\Tr{b}}
        {
            \Tr{c}
            \Tr{a}
        }
        ,
        \pstree[nodesep=1pt,levelsep=4ex]{\Tr{c}}
        {}
\end{Bmatrix}$}\\
%\end{array} \right. \] 
\\
$R$ contains rules,\\
%   $r_1=(\chi _1,(C_1|C_3),(C_2|C_4))$, $r_2=(\chi _2, (C_1|C_3),(C_2|C_4))$,  where \\
   $r_1=(\chi _1,U_1,U_2)$, $r_2=(\chi _2, U_1,U_2)$,  where \\
   $\chi _1=(a,-,-),\chi _2=(b,-,-)$, $U_1\in \{C_1,C_3\},U_2\in \{C_2,C_4\}$\\
   $C_1 =(A_1,a,1),C_2=(A_1,a,2),C_3=(A_2,V_1,1),C_4=(A_2,V_1,2)$, $V_1\in \{b,c\}$\\
%   $C_1 =(A_1,a,1),C_2=(A_1,a,2),C_3=(A_2,b|c,1),C_4=(A_2,b|c,2)$\\\\
%   $r_1=(\chi _1, ((A_1 | A_2),(a|b|c),1),((A_1|A_2),(a|b|c),2))$,  where $\chi _1=(a,-,-)$\\
%  $r_2=(\chi _2, ((A_1 | A_2),(a|b|c),1),((A_1|A_2),(a|b|c),2))$,  where $\chi _2=(b,-,-)$\\       
%   $r_1=(\chi _1, (V_1,V_2,1),(V_1,V_2,2))$ \\
%   $r_2=(\chi _2, (V_1,V_2,1),(V_1,V_2,2))$,  where\\
%   $\chi _1=(a,-,-)$, $\chi _2=(b,-,-)$\\ $V_1\in \{A_1,A_2\}$, $V_2\in \{a,b,c\}$\\        
   \noindent \rule{\textwidth}{1pt}
\end{example}   
\begin{example}
%\textbf{Example 2:} $L_{r_{12}}=\{t|\mbox{ no }a\mbox{ in } t\mbox{ has an } a \mbox{as child}.\}$\\\\        
                     $L_{r_{6}}=\{t|\mbox{ no }a\mbox{ in } t\mbox{ has an } a \mbox{ as child}.\}$\\\\  
$\Gamma _{r_{6}} = (\Sigma ,\mathcal{A},\mathcal{A} ',R)$ where,\\

    $\Sigma =\{a_2,b_2,c_0\}$,  
$\mathcal{A} = \{A_1\}$\\$F_1=[-1,-1,-1]$
$\mathcal{A}'= \{A_1\}$\\
{\small $A_1=
    \begin{Bmatrix}
        \pstree[nodesep=1pt,levelsep=4ex]{\Tr{c}}
        {}
        ,
        \pstree[nodesep=1pt,levelsep=4ex]{\Tr{a}}
        {
            \Tr{c}
            \Tr{c}
        }
        ,
        \pstree[nodesep=1pt,levelsep=4ex]{\Tr{a}}
        {
            \Tr{b}
            \Tr{b}
        }
        , 
        \pstree[nodesep=1pt,levelsep=4ex]{\Tr{a}}
        {
            \Tr{b}
            \Tr{c}
        }
        ,       
        \pstree[nodesep=1pt,levelsep=4ex]{\Tr{a}}
        {
            \Tr{c}
            \Tr{b}
        }
        ,
        \pstree[nodesep=1pt,levelsep=4ex]{\Tr{b}}
        {
            \Tr{c}
            \Tr{c}
        }
        ,
        \pstree[nodesep=1pt,levelsep=4ex]{\Tr{b}}
        {
            \Tr{a}
            \Tr{a}
        }
        ,
        \pstree[nodesep=1pt,levelsep=4ex]{\Tr{b}}
        {
            \Tr{b}
            \Tr{b}
        }
        ,\\ 
        \pstree[nodesep=1pt,levelsep=4ex]{\Tr{b}}
        {
            \Tr{a}
            \Tr{c}
        }
        ,
        \pstree[nodesep=1pt,levelsep=4ex]{\Tr{b}}
        {
            \Tr{c}
            \Tr{a}
        }       ,
        \pstree[nodesep=1pt,levelsep=4ex]{\Tr{b}}
        {
            \Tr{b}
            \Tr{c}
        }
        ,       
        \pstree[nodesep=1pt,levelsep=4ex]{\Tr{b}}
        {
            \Tr{c}
            \Tr{b}
        }
        ,
        \pstree[nodesep=1pt,levelsep=4ex]{\Tr{b}}
        {
            \Tr{a}
            \Tr{b}
        }
        ,
        \pstree[nodesep=1pt,levelsep=4ex]{\Tr{b}}
        {
            \Tr{b}
            \Tr{a}
        }
     \end{Bmatrix}$}\\
$R$ contains rules,\\\\
   $r_1=(\chi _1,C_1,C_2)$,$r_2=(\chi _2, C_3,C_4)$,  where \\
   $\chi _1=(a,-,-),\chi _2=(b,-,-)$\\        
    $C_1 =(A_1,V_1,1),C_2=(A_1,V_1,2),C_3=(A_1,V_2,1),C_4=(A_1,V_2,2)$\\
        $V_1\in \{c,b\},V_2\in \{a,b,c\}$  \\\\
\noindent \rule{\textwidth}{1pt}
\end{example}    
\begin{example}
%\textbf{Example 4:} $L_{r_{11}}=\{t|n_a(t)\%2\neq 0\}$\\\\        
                    $L_{r_{4}}=\{t|n_a(t)\%2\neq 0\}$\\\\  
$\Gamma _{r_{4}} = (\Sigma ,\mathcal{A},\mathcal{A} ',R)$ where,\\
$\Sigma =\{a_2,b_2,c_0\}$,  
$\mathcal{A} = \{A_1,A_2\}$,$F_1=[-1,-1,-1],F_2=[0,1,2]$,
$\mathcal{A}'= \{A_1,A_2\}$, where\\
{\small $A_1=
\begin{Bmatrix}
   %    \pstree[nodesep=1pt,levelsep=4ex]{\Tr{b}}
    %    {
     %       \Tr{c}
      %      \Tr{c}
       % }
        % ,
        \pstree[nodesep=1pt,levelsep=4ex]{\Tr{b}}
        {
            \Tr{b}
            \Tr{b}
        }
        , 
        \pstree[nodesep=1pt,levelsep=4ex]{\Tr{b}}
        {
            \Tr{b}
            \Tr{c}
        }
        ,       
        \pstree[nodesep=1pt,levelsep=4ex]{\Tr{b}}
        {
            \Tr{c}
            \Tr{b}
        }
        ,
        \pstree[nodesep=1pt,levelsep=4ex]{\Tr{b}}
        {}
        ,
        \pstree[nodesep=1pt,levelsep=4ex]{\Tr{c}}
        {}
\end{Bmatrix}$}\\
{\small $A_2=
    \begin{Bmatrix}
        \pstree[nodesep=1pt,levelsep=4ex]{\Tr{a}}
        {
            \Tr{c}
            \Tr{c}
        }
        ,
        \pstree[nodesep=1pt,levelsep=4ex]{\Tr{a}}
        {
            \Tr{b}
            \Tr{b}
        }
        , 
        \pstree[nodesep=1pt,levelsep=4ex]{\Tr{a}}
        {
            \Tr{b}
            \Tr{c}
        }
        ,       
        \pstree[nodesep=1pt,levelsep=4ex]{\Tr{a}}
        {
            \Tr{c}
            \Tr{b}
        }
        ,
        \pstree[nodesep=1pt,levelsep=4ex]{\Tr{a}}
        {
            \Tr{a}
            \Tr{a}
        }
        ,
        \pstree[nodesep=1pt,levelsep=4ex]{\Tr{b}}
        {
            \Tr{a}
            \Tr{c}
        }
        ,
        \pstree[nodesep=1pt,levelsep=4ex]{\Tr{b}}
        {
            \Tr{b}
            \Tr{a}
        }
        ,       
        \pstree[nodesep=1pt,levelsep=4ex]{\Tr{b}}
        {
            \Tr{a}
            \Tr{b}
        }
        ,
        \pstree[nodesep=1pt,levelsep=4ex]{\Tr{b}}
        {
            \Tr{c}
            \Tr{a}
        }
        ,
            \begin{comment}
        \pstree[nodesep=1pt,levelsep=4ex]{\Tr{b}}
        {
            \Tr{a}
            {\psset{linestyle=none}\TR{}}
        }
        ,
        \pstree[nodesep=1pt,levelsep=4ex]{\Tr{b}}
        {
            {\psset{linestyle=none}\TR{}}
            \Tr{a}
        }
        ,
        \pstree[nodesep=1pt,levelsep=4ex]{\Tr{a}}
        {
            \Tr{b}
            {\psset{linestyle=none}\TR{}}
        }
        ,
        \pstree[nodesep=1pt,levelsep=4ex]{\Tr{a}}
        {
            {\psset{linestyle=none}\TR{}}
            \Tr{b}
        }
        ,
        \pstree[nodesep=1pt,levelsep=4ex]{\Tr{a}}
        {
            \Tr{c}
            {\psset{linestyle=none}\TR{}}
        }
        ,
        \pstree[nodesep=1pt,levelsep=4ex]{\Tr{a}}
        {
            {\psset{linestyle=none}\Tr{}}
            \Tr{c}
        }
        ,
            \end{comment}
        \pstree[nodesep=1pt,levelsep=4ex]{\Tr{a}}
        {}
\end{Bmatrix}$}\\
\\
$R$ contains rules,\\
   $r_1=(\chi _1,U_1,U_2))$, $r_2=(\chi _2, U_1,U_2)$, where\\
%  $r_1=(\chi _1,(C_1|C_3),(C_2|C_4))$\\ $r_2=(\chi _2, (C_1|C_3),(C_2|C_4))$\\
  %$r_3=(\chi _3,(C_1|C_4))$\\ $r_4=(\chi _4,(C_1|C_3))$\\  
  %$r_5=(\chi _5,(C_2|C_4))$\\ $r_6=(\chi _6, (C_1|C_3))$\\
  %$r_7=(\chi _7,(C_2|C_4))$\\ $r_8=(\chi _8, (C_1|C_3))$ \\
  %$r_9=(\chi _9,(C_2|C_4))$\\ $r_{10}=(\chi _{10}, (C_1|C_3))$,   where \\
   $\chi _1=(a,-,-),\chi _2=(b,-,-)$, $U_1\in\{C_1,C_3\},U_2\in\{C_2,C_4\}$\\
   %\\,\chi _3=(b,a,-),\chi _4=(b,-,a)$, $\chi _5=(a,b,-)$\\
  % $\chi _6=(a,-,b),\chi _7=(a,c,-),\chi _8=(a,-,c),\chi _9=(b,c,-)$, $\chi _{10}=(b,-,c)$\\
    $C_1 =(A_1,V_1,1),C_2=(A_1,V_1,2),C_3=(A_2,V_2,1),C_4=(A_2,V_2,2)$, $V_1\in\{c,b\},V_2\in\{a,b\}$\\
\noindent \rule{\textwidth}{1pt}
\end{example}    

\begin{example}
%\textbf{Example 4:} $L_{r_{11}}=\{t|n_a(t)\%2\neq 0\}$\\\\        
                    $L_{r_{4}}=\{t|(n_a(t)\%3)= 0 \mbox{ and } (n_b(t)\%2)= 0\}$\\\\  
$\Gamma _{r_{4}} = (\Sigma ,\mathcal{A},\mathcal{A} ',R)$ where,\\
$\Sigma =\{a_2,b_2,c_0\}$,  
$\mathcal{A} = \{A_1,A_2,A_3\}$,$F_1=[0,0,1],F_2=[0,0,2],F_3=[-1,-1,-1]$,
$\mathcal{A}'= \{A_1,A_2\}$, where\\
{\small $A_1=
\begin{Bmatrix}
        \pstree[nodesep=1pt,levelsep=4ex]{\Tr{b}}
        {}
\end{Bmatrix}$},
{\small $A_2=
    \begin{Bmatrix}
        \pstree[nodesep=1pt,levelsep=4ex]{\Tr{a}}
        {}
\end{Bmatrix}$},
{\small $A_3=
    \begin{Bmatrix}
	\pstree[nodesep=1pt,levelsep=4ex]{\Tr{a}}
        {
            \Tr{a}
            \Tr{a}
        }
	,
        \pstree[nodesep=1pt,levelsep=4ex]{\Tr{b}}
        {
            \Tr{c}
            \Tr{b}
        }
        , 
        \pstree[nodesep=1pt,levelsep=4ex]{\Tr{b}}
        {
            \Tr{b}
            \Tr{c}
        }
	,
        \pstree[nodesep=1pt,levelsep=4ex]{\Tr{c}}
        {}
\end{Bmatrix}$}\\
\\
$R$ contains rules,\\
   $r_1=(\chi _1,U_1,U_2))$, $r_2=(\chi _2, U_1,U_2)$, where\\
%  $r_1=(\chi _1,(C_1|C_3),(C_2|C_4))$\\ $r_2=(\chi _2, (C_1|C_3),(C_2|C_4))$\\
  %$r_3=(\chi _3,(C_1|C_4))$\\ $r_4=(\chi _4,(C_1|C_3))$\\  
  %$r_5=(\chi _5,(C_2|C_4))$\\ $r_6=(\chi _6, (C_1|C_3))$\\
  %$r_7=(\chi _7,(C_2|C_4))$\\ $r_8=(\chi _8, (C_1|C_3))$ \\
  %$r_9=(\chi _9,(C_2|C_4))$\\ $r_{10}=(\chi _{10}, (C_1|C_3))$,   where \\
   $\chi _1=(a,-,-),\chi _2=(b,-,-)$, $U_1\in\{C_1,C_3\},U_2\in\{C_2,C_4\}$\\
   %\\,\chi _3=(b,a,-),\chi _4=(b,-,a)$, $\chi _5=(a,b,-)$\\
  % $\chi _6=(a,-,b),\chi _7=(a,c,-),\chi _8=(a,-,c),\chi _9=(b,c,-)$, $\chi _{10}=(b,-,c)$\\
    $C_1 =(A_1,V_1,1),C_2=(A_1,V_1,2),C_3=(A_2,V_2,1),C_4=(A_2,V_2,2)$, $V_1\in\{c,b\},V_2\in\{a,b\}$\\
\noindent \rule{\textwidth}{1pt}
\end{example}    
\begin{example}
%\textbf{Example 4:} $L_{r_{11}}=\{t|n_a(t)\%2\neq 0\}$\\\\        
                    $L_{r_{4}}=\{t|((n_a(t)-2)\%3)= 0\}$\\\\  
$\Gamma _{r_{4}} = (\Sigma ,\mathcal{A},\mathcal{A} ',R)$ where,\\
$\Sigma =\{a_2,b_2,c_0\}$,  
$\mathcal{A} = \{A_1,A_2\}$,$F_1=[-1,-1,-1],F_2=[0,2,4]$,
$\mathcal{A}'= \{A_1,A_2\}$, where\\
{\small $A_1=
\begin{Bmatrix}
   %    \pstree[nodesep=1pt,levelsep=4ex]{\Tr{b}}
    %    {
     %       \Tr{c}
      %      \Tr{c}
       % }
        % ,
        \pstree[nodesep=1pt,levelsep=4ex]{\Tr{b}}
        {
            \Tr{b}
            \Tr{b}
        }
        , 
        \pstree[nodesep=1pt,levelsep=4ex]{\Tr{b}}
        {
            \Tr{b}
            \Tr{c}
        }
        ,       
        \pstree[nodesep=1pt,levelsep=4ex]{\Tr{b}}
        {
            \Tr{c}
            \Tr{b}
        }
        ,
        \pstree[nodesep=1pt,levelsep=4ex]{\Tr{b}}
        {}
        ,
        \pstree[nodesep=1pt,levelsep=4ex]{\Tr{c}}
        {}
\end{Bmatrix}$}\\
{\small $A_2=
    \begin{Bmatrix}
        \pstree[nodesep=1pt,levelsep=4ex]{\Tr{a}}
        {
            \Tr{c}
            \Tr{c}
        }
        ,
        \pstree[nodesep=1pt,levelsep=4ex]{\Tr{a}}
        {
            \Tr{b}
            \Tr{b}
        }
        , 
        \pstree[nodesep=1pt,levelsep=4ex]{\Tr{a}}
        {
            \Tr{b}
            \Tr{c}
        }
        ,       
        \pstree[nodesep=1pt,levelsep=4ex]{\Tr{a}}
        {
            \Tr{c}
            \Tr{b}
        }
        ,
        \pstree[nodesep=1pt,levelsep=4ex]{\Tr{a}}
        {
            \Tr{a}
            \Tr{a}
        }
        ,
        \pstree[nodesep=1pt,levelsep=4ex]{\Tr{b}}
        {
            \Tr{a}
            \Tr{c}
        }
        ,
        \pstree[nodesep=1pt,levelsep=4ex]{\Tr{b}}
        {
            \Tr{b}
            \Tr{a}
        }
        ,       
        \pstree[nodesep=1pt,levelsep=4ex]{\Tr{b}}
        {
            \Tr{a}
            \Tr{b}
        }
        ,
        \pstree[nodesep=1pt,levelsep=4ex]{\Tr{b}}
        {
            \Tr{c}
            \Tr{a}
        }
        ,
        \pstree[nodesep=1pt,levelsep=4ex]{\Tr{a}}
        {}
\end{Bmatrix}$}\\
\\
$R$ contains rules,\\
   $r_1=(\chi _1,U_1,U_2))$, $r_2=(\chi _2, U_1,U_2)$, where\\
%  $r_1=(\chi _1,(C_1|C_3),(C_2|C_4))$\\ $r_2=(\chi _2, (C_1|C_3),(C_2|C_4))$\\
  %$r_3=(\chi _3,(C_1|C_4))$\\ $r_4=(\chi _4,(C_1|C_3))$\\  
  %$r_5=(\chi _5,(C_2|C_4))$\\ $r_6=(\chi _6, (C_1|C_3))$\\
  %$r_7=(\chi _7,(C_2|C_4))$\\ $r_8=(\chi _8, (C_1|C_3))$ \\
  %$r_9=(\chi _9,(C_2|C_4))$\\ $r_{10}=(\chi _{10}, (C_1|C_3))$,   where \\
   $\chi _1=(a,-,-),\chi _2=(b,-,-)$, $U_1\in\{C_1,C_3\},U_2\in\{C_2,C_4\}$\\
   %\\,\chi _3=(b,a,-),\chi _4=(b,-,a)$, $\chi _5=(a,b,-)$\\
  % $\chi _6=(a,-,b),\chi _7=(a,c,-),\chi _8=(a,-,c),\chi _9=(b,c,-)$, $\chi _{10}=(b,-,c)$\\
    $C_1 =(A_1,V_1,1),C_2=(A_1,V_1,2),C_3=(A_2,V_2,1),C_4=(A_2,V_2,2)$, $V_1\in\{c,b\},V_2\in\{a,b\}$\\
\noindent \rule{\textwidth}{1pt}
\end{example}    
%\emph{Note: In examples 3, $A_1=\{a\}$ and in example 4, $A_2=\{a\}$ will also works correctly. But we need to change the rules accordingly.}
\subsection{Equivalence with Regular Tree Grammar}
\emph{Tree grammars} are generating devices which is used for generating trees.
The language generated by regular tree grammar is called \emph{regular tree language}.

\emph{Tree automata} are accepting devices for trees.
Finite tree automata are generalizations of word automata. While a word automaton accepts
a word, a tree automaton accepts a tree.
% According to the manner in which the automaton runs on the input tree, 
Finite tree automata can be either bottom-up or top-down~\cite{hubert}. A \emph{top-down tree automaton} starts its computation at
the root of the tree and then simultaneously works down the paths of the tree level by level.
%The tree automaton accepts the tree if such a run can be defined. 
A \emph{bottom-up tree automaton}
starts its computation in the leaves of the input tree and works its way up towards the root.
A finite tree automaton can be either deterministic or non-deterministic. 
%This is an important issue since
Deterministic top-down automata are strictly less expressive than non-deterministic
top-down automata. For the bottom-up case, deterministic bottom-up tree automata are just
as powerful, from the point of view of language equivalence, as non-deterministic bottom-up
tree automata. Non-deterministic top-down tree automata are equivalent to non-deterministic
bottom-up tree automata.
%\newpage
\subsection{Insertion System to Regular Tree Grammar}
\subsubsection{Method of conversion}
For a given tree insertion system $\Gamma=(\Sigma,\mathcal{A} ,\mathcal{A} ',R)$ we can construct an equivalent regular tree grammar 
\hbox{ $G=(N,T,S,\bar{P})$}.

First we consider the simple case where the flag $F_i=[-1,-1,-1], \forall A_i\in\mathcal{A}$ 
\begin{itemize}
	\item $T=\Sigma$
	\item $N$ contains the start symbol  $S$ initially and more symbols of the form $M',M^{2'},\cdots , M^{k'}$ for some fixed $k$, are added to $N$ 
		as we proceed to define the rules. 
	\item 	$\forall A_i\in\mathcal{A}'$, if $t\in A_i$ with root with label $p$ having arity $m>0$ and children with label $p_1,p_2,\cdots ,p_m$, 
		then \hbox{$S\rightarrow p(P_1,P_2,\cdots ,P_m)$} 
		will be a production rule and if there is a node in $t$ with label $q$ and arity $r$  having children with labels $q_1,q_2,\cdots ,q_r$, 
		then \hbox{$Q\rightarrow q(Y_1,Y_2,\cdots ,Y_r)$} will be a production rule where $Y_i=Q_i, \forall q_i$ with label not equal to $q$ 
		and $Y_i=Q_i', \forall q_i$ with label equal to $q$. \hbox{$Q_i'\rightarrow q(Y_1,Y_2,\cdots ,Y_r)$} with \hbox{$Y_i=Q_i^{2'},
		\forall Y_i=Q_i$}.

		In general \hbox{$Q_i^{n'}\rightarrow q(Y_1,Y_2,\cdots ,Y_r)$} 	with  $Y_i=Q_i^{(n+1)'}, \forall Y_i=Q_i^{n'}$. 
		
		If $m=0$, \hbox{$S\rightarrow P$} is a production rule.

	\item	$\forall r_i\in R, r_i=(\chi,C_1,C_2,\cdots ,C_k)$ where $\chi = (p,left,right)$, $C_j=(X_j,rt_j,k'), \forall j\leq k$ with 
		$rt_j\in \Sigma , X_j\in \mathcal{A}$, \hbox{$P^{n'}\rightarrow p(RT_j)$}, where $RT_j$ is the non terminal corresponding to $rt_j$, 
		is a production rule.
		
	\item	$\forall t\in \mathcal{A}-\mathcal{A'}$ with  root node having label $p$ and arity 0, then \hbox{$P\rightarrow p$} is a production rule.

	\item	$N$ will includes all such $P_i$ , $Q_i$ and $Q_i'$.
\end{itemize}
It can easily be proved by induction that $L(G)=L(\Gamma)$.
\begin{example}
%\textbf{Example 1:} $L_{r_1}=\{a(b^i(g),c^j(g)),i,j\ge 1\}$\\\\
$L_{r_1}=\{a(b^i(h),c^j(h)),i,j\ge 1\}$\\\\       
$\Gamma _{r_1}= (\Sigma ,\mathcal{A},\mathcal{A} ',R)$ where,\\
$\Sigma =\{a_2,b_1,c_1,h_0\}$, 
$\mathcal{A}=\{A_1,A_2\}$\\$F_1=\{-1,-1,-1\},F_2=\{-1,-1,-1\}$
$\mathcal{A} '=\{A_1\}$, where \\
$A_1=
\begin{Bmatrix}
        \pstree[nodesep=1pt,levelsep=3ex]{\Tr{a}}
        {
            \Tr{b}
            \Tr{c}
        }
        ,
        \pstree[nodesep=1pt,levelsep=3ex]{\Tr{a}}
        {
            \pstree[nodesep=1pt,levelsep=3ex]{\Tr{b}}
            {
                \Tr{h}
            }
            \pstree[nodesep=1pt,levelsep=3ex]{\Tr{c}}
            {
                \Tr{h}
            }
        }

\end{Bmatrix}
,
\hspace{3mm}
A_2=
\begin{Bmatrix}
        \pstree[nodesep=1pt,levelsep=6ex]{\Tr{b}}
        {}
        \hspace{2mm}
        ,
        \hspace{2mm}
        \pstree[nodesep=1pt,levelsep=6ex]{\Tr{c}}
        {}
        \hspace{2mm}
        ,
        \hspace{2mm}
        \pstree[nodesep=1pt,levelsep=6ex]{\Tr{h}}
        {}

\end{Bmatrix}$
\\\\
$R=\{r_1,r_2,r_3,r_4\}$, where\\\\
   $r_1=(\chi _1, C_1)$,  where $\chi _1=(b,-)$ and $C_1= (A_2,b,1)$ \\
   $r_2=(\chi _2, C_2)$,  where $\chi _2=(c,-)$ and $C_2= (A_2,c,1)$ \\
   $r_3=(\chi _1, C_3)$   where $C_3=(A_2,h,1)$\\
   $r_4=(\chi _2, C_3)$\\  
\\The production rules are\\
$ \bar{P}=\{
     S \rightarrow  a(B,C),
B \rightarrow  b(B)|b(H) , 
C \rightarrow  c(C)|c(H),
H \rightarrow  h  \}$

   The regular tree grammar corresponding to $\Gamma _{r_2}$ is    $G_{r_2}=(\{S,B,C,H\},\{a,b,c,h\},S,\bar{P})$\\
\noindent \rule{\textwidth}{1pt}
\end{example} 
\begin{example}
$L_{r_2}=\{a(b^i(h),c^j(h)),i,j\ge 1\mbox{, } i\%2 =0\}$\\\\
$\Gamma _{r_2} = (\Sigma,\mathcal{A},\mathcal{A} ',R)$ where,\\
$\Sigma=\{a_2,b_1,c_1,h_0\}$,  
$\mathcal{A} = \{A_1,A_2\}$,$F_1=[-1,-1,-1],F_2=[-1,-1,-1]$,
$\mathcal{A}'= \{A_1\}$, where\\
{\small
$A_1=
\begin{Bmatrix}
        \pstree[nodesep=1pt,levelsep=3ex]{\Tr{a}}
        {
            \pstree[nodesep=1pt,levelsep=3ex]{\Tr{b}}
            {
                \Tr{b}
            }
            \Tr{c}
        }
        ,
        \pstree[nodesep=1pt,levelsep=3ex]{\Tr{a}}
        {
            \pstree[nodesep=1pt,levelsep=3ex]{\Tr{b}}
            {
                \pstree[nodesep=1pt,levelsep=3ex]{\Tr{b}}
                {
                  \Tr{h}
                }
            }
            \pstree[nodesep=1pt,levelsep=3ex]{\Tr{c}}
            {
                \Tr{h}
            }
        }
 \end{Bmatrix}
 ,
\hspace{3mm}
A_2=
\begin{Bmatrix}
        \pstree[nodesep=1pt,levelsep=4ex]{\Tr{b}}
        {
            \Tr{b}
        }
         \hspace{2mm}
        ,
        \hspace{2mm}
        \pstree[nodesep=1pt,levelsep=6ex]{\Tr{c}}
        {}
        \hspace{2mm}
        ,
        \hspace{2mm}
        \pstree[nodesep=1pt,levelsep=6ex]{\Tr{h}}
        {}
\end{Bmatrix}$}
\\\\
$R=\{r_1,r_2,r_3,r_4\}$, where\\\\
   $r_1=(\chi _1, C_1)$,  $r_2=(\chi _2, C_2)$, $r_3=(\chi _1, C_3)$, $r_4=(\chi _2, C_3)$, where \\
$\chi _1=(b,-)$,  $\chi _2=(c,-)$,  $C_1= (A_2,b,1)$, $C_2= (A_2,c,1)$, $C_3=(A_2,h,1)$\\
\\The production rules are\\
$ \bar{P}=\{
     S \rightarrow  a(B,C),
B \rightarrow  b(B'),
B' \rightarrow  b(B)|b(H) , 
C \rightarrow  c(C)|c(H),
H \rightarrow  h  \}$

   The regular tree grammar corresponding to $\Gamma _{r_2}$ is    $G_{r_2}=(\{S,B,B',C,H\},\{a,b,c,h\},S,\bar{P})$\\
\noindent \rule{\textwidth}{1pt}
\end{example}    
\begin{example}
 $L_{r_5}=\{a(b^i(h),c^j(h)),i,j\ge 1\mbox{, } i\%3=0,j\%2 =0\}$\\\\
$\Gamma _{r_5} = (F,\mathcal{A},\mathcal{A} ',R)$ where,\\
$F=\{a_2,b_1,c_1,h_0\}$,  
$\mathcal{A} = \{A_1^{[-1,-1,-1]},A_2^{[-1,-1,-1]}\}$,
$\mathcal{A}'= \{A_1\}$, where\\
$A_1=
\begin{Bmatrix}
        \pstree[nodesep=1pt,levelsep=3ex]{\Tr{a}}
        {
            \pstree[nodesep=1pt,levelsep=3ex]{\Tr{b}}
            {
                \pstree[nodesep=1pt,levelsep=3ex]{\Tr{b}}
                {
                    \Tr{b}
                }
            }
            \pstree[nodesep=1pt,levelsep=3ex]{\Tr{c}}
            {
                \Tr{c}
            }
        }
        ,
        \pstree[nodesep=1pt,levelsep=3ex]{\Tr{a}}
        {
            \pstree[nodesep=1pt,levelsep=3ex]{\Tr{b}}
            {
                \pstree[nodesep=1pt,levelsep=3ex]{\Tr{b}}
                {
                    \pstree[nodesep=1pt,levelsep=3ex]{\Tr{b}}
                    {
                        \Tr{h}
                    }
                }
            }
            \pstree[nodesep=1pt,levelsep=3ex]{\Tr{c}}
            {
                \pstree[nodesep=1pt,levelsep=3ex]{\Tr{c}}
                {
                    \Tr{h}
                }
            }
        }
 \end{Bmatrix}
 ,
\hspace{3mm}
A_2=
\begin{Bmatrix}
        \pstree[nodesep=1pt,levelsep=4ex]{\Tr{b}}
        {
            \pstree[nodesep=1pt,levelsep=4ex]{\Tr{b}}
            {
                \Tr{b}
            }
        }
         \hspace{2mm}
        ,
        \hspace{2mm}
        \pstree[nodesep=1pt,levelsep=6ex]{\Tr{c}}
        {
            \Tr{c}
        }
        \hspace{2mm}
        ,
        \hspace{2mm}
        \pstree[nodesep=1pt,levelsep=6ex]{\Tr{h}}
        {}
\end{Bmatrix}$
\\\\
$R=\{r_1,r_2,r_3,r_4\}$, where\\\\
   $r_1=(\chi _1, C_1)$,  where $\chi _1=(b,-)$ and $C_1= (A_2,b,1)$ \\
   $r_2=(\chi _2, C_2)$,  where $\chi _2=(c,-)$ and $C_2= (A_2,c,1)$ \\
   $r_3=(\chi _1, C_3)$   where $C_3=(A_2,h,1)$\\
   $r_4=(\chi _2, C_3)$\\  
\\The production rules are\\
$ \bar{P}=\{
     S \rightarrow  a(B,C),
B \rightarrow  b(B'),
B' \rightarrow  b(B''),
B'' \rightarrow  b(B)|b(H) ,
C \rightarrow  c(C'), 
C' \rightarrow  c(C)|c(H),
H \rightarrow  h  \}$

   The regular tree grammar corresponding to $\Gamma _{r_2}$ is    $G_{r_2}=(\{S,B,B',B'',C,H\},\{a,b,c,h\},S,\bar{P})$\\
\noindent \rule{\textwidth}{1pt}
\end{example}    


The above case will not take care of languages like trees with node labels $a$, $b$, and $c$ where number of $a's$ is odd. For such cases we 
give the construction below.
\begin{itemize}
	\item $T=\Sigma$
	\item $N$ contains the start symbol  $S$ initially and more symbols of the form $M',M^{2'},\cdots , M^{k'}$ for some fixed $k$, are added to $N$ 
		as we proceed to define the rules. 
	\item Let $AJ\subseteq \mathcal{A}'$, where for each $A_i \in AJ, x_i=y_i=z_i \mbox{ in }F_i$ and 
		$AI= (\mathcal{A}'-AJ)$, where for each $A_i\in AI,$\hbox{$ n_i=|x_i-y_i|.$} (This is used to take care of some constraint on the number of 
		a particular terminal symbol $\sigma_k\in \Sigma$).
	\item $\forall A_i\in \mathcal{A}$, if there is a node with label $p$ arity 0 is in $A_i$, then \hbox{$P\rightarrow p$} is a production rule.
	\item $\forall A_i\in AJ$\\
		%If $\exists A_l\in AI\mbox{ with }(x_l\neq y_l)\mbox{ in } F_i$, 
		If  $\exists A_l\in AI\mbox{ with }(x_l\neq y_l)\mbox{ in } F_l$, $t\in A_i$ with root node with label $p$ 
		having arity $m$ and children with label 
		$p_1,p_2,\cdots ,p_{m}$, where for  some $j\leq m$, $p_j$ is the label of root node of some $t'\in A_l$, then from 
		\hbox{$S\rightarrow p(P_1,P_2,\cdots ,P_m)$}  write  $j$ production rules, with $P_k=S,\forall k\leq j$.\\
		If $m=0$, \hbox{$S\rightarrow P$} will be a production rule.\\

		$\forall A_l'\in AI$, if $(x_l'= y_l')\mbox{ in }F_l'$ then \hbox{$S\rightarrow p(P_1,P_2,\cdots ,P_m)$} will also be a production rule.

		$N$ will includes all such $P_i$.

	\item $\forall A_i\in AI$ with $n_i=1$\\
		If $t\in A_i$ with root with label $p$ having arity $m$ and children with label $p_1,p_2,\cdots ,p_m$, 
		then \hbox{$S\rightarrow p(P_1,P_2,\cdots ,P_m)$} 
		will be a production rule and  if there is a node in $t$ with label $q$ and arity $r$  having children with labels $q_1,q_2,\cdots ,q_r$,
		then, \hbox{$Q\rightarrow q(Q_1,Q_2,\cdots ,Q_r)$} will be a production rule.\\If $m=0$, \hbox{$S\rightarrow P$} will be a production rule.
				
		$\forall r_i\in R, r_i=(\chi,C_1,C_2,\cdots ,C_{k'})$ where $\chi = (p,left,right)$, $C_j=(X_j,rt_j,k''), \forall j\leq k'$ with 
		$rt_j\in \Sigma , X_j\in \mathcal{A}$
	
		If $\forall j\leq k', X_j\notin AI$, \hbox{$P\rightarrow p(RT_1,RT_2,\cdots ,RT_k)$} will be a production rule, 

		If $\exists X_j\in AI,j\leq k$, 
			\begin{itemize}
			\item   $P\rightarrow p(Y_1,Y_2,\cdots ,Y_k)$ will be a production rule with 
				$Y_j=RT_j, \forall j \mbox{ where }X_j\notin AI$ and \hbox{$Y_j=RT'_j, \forall j \mbox{ where } X_j\in AI$.}

		%	If $\forall k, Y_k=RT_{j}'$ for some $rt_{j}$, then \hbox{$P\rightarrow p(RT_1,RT_2,\cdots ,RT_k)$}
		%	is a production rule.				

				If $rt_j=\sigma _j$ where $X_j\in AI$ then \hbox{$RT'_{\sigma _k}\rightarrow rt_j(Y_1,Y_2,\cdots ,Y_{p'})$} 
				will be a production rule with $ Y_{p_{''}}=S$ for some $p''\leq p'$, where $p'$ is the arity of $rt_j$.

				If $rt_j\neq \sigma _k$ where $X_j\in AI$ then \hbox{$RT'_j\rightarrow rt_j(Y_1,Y_2,\cdots ,Y_{p'})$} 
				will be a production rule  $Y_{p_{''}}=RT'_{\sigma _k}$ for some $p^{''}\leq p'$.

			\end{itemize}
		%f $\forall k, Y_k=RT_{j}'$  for some $rt_{j} $in  \hbox{$P\rightarrow p(Y_1,Y_2,\cdots ,Y_r)$}, 
		%hen \hbox{$P\rightarrow p(RT_1,RT_2,\cdots ,RT_k)$} is a production rule.				

		$N$ will includes all such $P_i$, $Q_i$, $RT_i'$ and $Y_i$.
	\item   $\forall A_i\in AI$ with $n_i\neq 1$\\
		If $t\in A_i$ with root with label  $p\neq \sigma _i$ having arity $m$ and children with label  $p_1,p_2,\cdots ,p_m$, 
		with some $p_j =\sigma _k,j\leq m$
		then  \hbox{$S\rightarrow p(P_1,P_2,\cdots ,P_m)$} , where both $P_j=P_{\sigma_k}$ and $P_j=S$ are production rules. Then 
		\hbox{$P_{\sigma_k}\rightarrow S$}  and \hbox{$P_{\sigma_k}\rightarrow p_{\sigma_k}(Y_1,Y_2,\cdots , Y_{m'})$} will be production rules 
		where $m'$ is the arity of node with label $p_{\sigma_k}$ and for some 
		$m''\leq m', Y_{m''}=P_{\sigma_k}'$. \hbox{$P_{\sigma_k}'\rightarrow p_{\sigma_k}(Y_1,Y_2,\cdots , Y_{m'})$} and for some
		$m''\leq m', Y_{m''}=P_{\sigma_k}^{2'}$. In general \hbox{$P_{\sigma_k}^{q'}\rightarrow p_{\sigma_k}(Y_1,Y_2,\cdots , Y_{m'})$} and for some
		$m''\leq m', Y_{m''}=P_{\sigma_k}^{(q+1)'}$ and  \hbox{$P_{\sigma_k}^{z_i'}\rightarrow p_{\sigma_k}(Y_1,Y_2,\cdots , Y_{m'})$}
		and for some $m''\leq m', Y_{m''}=S$.

		If $m=0$, \hbox{$S\rightarrow P$} will be a production rule.

		If $t\in A_i$ with root node with label $p= \sigma _k$ having arity $m$ and children with label $p_1,p_2,\cdots ,p_m$,
		with some $p_j ,j\leq m$ is the label of root node of some $t'\in AI$, then \hbox{$S\rightarrow p(P_1,P_2,\cdots ,P_m)$}  
		will be a  production rule with $P_j=P_j'$ and \hbox{$P_j'\rightarrow p_j(Y_1,Y_2,\cdots , Y_{m'})$} where $m'$ is the arity of 
		node with label $p_j$ and for some $m''\leq m', Y_{m''}=P_{\sigma_k}'$.


	 	$\forall r_i\in R, r_i=(\chi,C_1,C_2,\cdots ,C_k')$ where $\chi = (p,left,right)$, $C_j=(X_j,rt_j,k''), \forall j\leq k'$ with 
		$rt_j\in \Sigma , X_j\in \mathcal{A}$
	
		
		If $\forall j, X_j\in AJ$, with $p\neq \sigma_k$ then \hbox{$P\rightarrow p(RT_1,RT_2,\cdots ,RT_m)$} will be a production rule where
		$m$ is the arity of node with label $p$.
		
		If $\forall j, X_j\in AI$, if $p\neq \sigma_k$ then \hbox{$P\rightarrow p(RT'_1,RT'_2,\cdots ,RT'_m)$} will be a production rule where
		$m$ is the arity of node with label $p$.

		If $p=\sigma_k$ then \hbox{$P\rightarrow p(Y_1,Y_2,\cdots ,Y_k)$} is a production rule with $Y_j=RT_j, \forall j$ where $X_j\in AJ$
		$Y_j=RT'_j, \forall j$ where $X_j\in AI$.
		
		If $rt_j=\sigma _k$ where $X_j\in AI$ then \hbox{$RT'_{\sigma _k}\rightarrow rt_j(Y_1,Y_2,\cdots ,Y_{p'})$} 
		will be a production rule with $Y_{p_{''}}=RT_{\sigma _k}^{2'}$ for some $p''\leq p'$, where $p'$ is the arity of $rt_j$.
					
		In general,  for some $p''\leq p$, \hbox{$RT^{{(z_j-1)}'}_{\sigma _k}\rightarrow rt_j(Y_1,Y_2,\cdots ,Y_{p'}),$} 
		with $Y_{p_{''}}=RT^{(z_i)'}_{\sigma _k}$, and \hbox{$RT^{z_i'}_{\sigma _k}\rightarrow rt_j(Y_1,Y_2,\cdots ,Y_{p'})$}
		, with 	$Y_{p_{''}}=S$. 
				
		If $rt_j\neq \sigma _k$ where $X_j\in AI$ then \hbox{$RT'_j\rightarrow rt_j(Y_1,Y_2,\cdots ,Y_{p'})$} 
		will be a production rule  $Y_{p_{''}}=RT'_{\sigma _k}$ for some $p^{''}\leq p'$.

		
		$N$ will includes all such $P_i's$, $RT_i's$ and $Y_i's$.
	\item   If $\forall k, Y_k=RT_{j}'$  for some $rt_{j} $in  \hbox{$P\rightarrow p(Y_1,Y_2,\cdots ,Y_r)$}, 
		then \hbox{$P\rightarrow p(RT_1,RT_2,\cdots ,RT_k)$} will be a production rule.				

\end{itemize}
It can easily be proved by induction that $L(G)=L(\Gamma)$.






\begin{example}
%\textbf{Example 1:} $L_{r_{10}}=\{t | t\mbox{ has  }a\mbox{ as root}\}$\\\\
                     $L_{r_{3}}=\{t | t\mbox{ has  }a\mbox{ as root}\}$\\\\

$\Gamma _{r_{3}} = (\Sigma ,\mathcal{A},\mathcal{A} ',R)$ where,\\

    $\Sigma =\{a_2,b_2,c_0\}$,  
$\mathcal{A} = \{A_1,A_2\}$, $F_1=[-1,-1,-1],F_2=[-1,-1,-1]$,
$\mathcal{A}'= \{A_1\}$\\


{\small $A_1=
    \begin{Bmatrix}
        \pstree[nodesep=1pt,levelsep=4ex]{\Tr{a}}
        {
            \Tr{c}
            \Tr{c}
        }
        ,
        \pstree[nodesep=1pt,levelsep=4ex]{\Tr{a}}
        {
            \Tr{a}
            \Tr{a}
        }
        ,
        \pstree[nodesep=1pt,levelsep=4ex]{\Tr{a}}
        {
            \Tr{b}
            \Tr{b}
        }
        , 
         \pstree[nodesep=1pt,levelsep=4ex]{\Tr{a}}
        {
            \Tr{a}
            \Tr{c}
        }
        ,
        \pstree[nodesep=1pt,levelsep=4ex]{\Tr{a}}
        {
            \Tr{c}
            \Tr{a}
        }       ,
        \pstree[nodesep=1pt,levelsep=4ex]{\Tr{a}}
        {
            \Tr{b}
            \Tr{c}
        }
        ,       
        \pstree[nodesep=1pt,levelsep=4ex]{\Tr{a}}
        {
            \Tr{c}
            \Tr{b}
        }
        ,
        \pstree[nodesep=1pt,levelsep=4ex]{\Tr{a}}
        {
            \Tr{a}
            \Tr{b}
        }
        ,
        \pstree[nodesep=1pt,levelsep=4ex]{\Tr{a}}
        {
            \Tr{b}
            \Tr{a}
        }
   \end{Bmatrix}$}\\


{\small
    $A_2=
\begin{Bmatrix}
       \pstree[nodesep=1pt,levelsep=4ex]{\Tr{c}}
        {}
        ,
        \pstree[nodesep=1pt,levelsep=4ex]{\Tr{b}}
        {
            \Tr{c}
            \Tr{c}
        }
        ,
        \pstree[nodesep=1pt,levelsep=4ex]{\Tr{b}}
        {
            \Tr{a}
            \Tr{a}
        }
        ,
        \pstree[nodesep=1pt,levelsep=4ex]{\Tr{b}}
        {
            \Tr{b}
            \Tr{b}
        }
        , 
         \pstree[nodesep=1pt,levelsep=4ex]{\Tr{b}}
        {
            \Tr{a}
            \Tr{c}
        }
        ,
        \pstree[nodesep=1pt,levelsep=4ex]{\Tr{b}}
        {
            \Tr{c}
            \Tr{a}
        }       ,
        \pstree[nodesep=1pt,levelsep=4ex]{\Tr{b}}
        {
            \Tr{b}
            \Tr{c}
        }
        ,       
        \pstree[nodesep=1pt,levelsep=4ex]{\Tr{b}}
        {
            \Tr{c}
            \Tr{b}
        }
        ,
        \pstree[nodesep=1pt,levelsep=4ex]{\Tr{b}}
        {
            \Tr{a}
            \Tr{b}
        }
        ,
        \pstree[nodesep=1pt,levelsep=4ex]{\Tr{b}}
        {
            \Tr{b}
            \Tr{a}
        }

\end{Bmatrix}$}\\

$R$ contains rules,\\\\
%   $r_1=(\chi _1,(C_1|C_3),(C_2|C_4))$, $r_2=(\chi _2, (C_1|C_3),(C_2|C_4))$,  where \\
   $r_1=(\chi _1,U_1,U_2)$, $r_2=(\chi _2, U_1,U_2)$,  where \\
   $\chi _1=(a,-,-),\chi _2=(b,-,-)$,        
   $U_1\in \{C_1,C_3\},U_2\in \{C_2,C_4\}$\\
   $C_1 =(A_1,a,1),C_2=(A_1,a,2),C_3=(A_2,V_1,1),C_4=(A_2,V_1,2)$\\
   $V_1\in \{b,c\}$\\\\
%   $C_1 =(A_1,a,1),C_2=(A_1,a,2),C_3=(A_2,b|c,1),C_4=(A_2,b|c,2)$\\
 The production rules are
\begin{eqnarray*}   
P=\{
S &\rightarrow & a(C,C)|a(B,B)|a(B,C)|a(C,B)|a(A,A)|a(A,B)|a(A,C)|a(B,A)|a(C,A)\\
A &\rightarrow & a(A,B)|a(A,C)|a(C,C)|a(C,A)|a(C,B)|a(B,C)|a(B,A)|a(A,A)|a(B,B)\\
B &\rightarrow & b(A,B)|b(A,C)|b(C,C)|b(C,A)|b(C,B)|b(B,C)|b(B,A)|b(A,A)|b(B,B)\\
C &\rightarrow & c\}
\end{eqnarray*}
  The regular tree grammar corresponding to $\Gamma _{r_{3}}$ is 
  $G_{r_3}=(\{S,A,B,C\},\{a,b,c\},S,P)$

\noindent \rule{\textwidth}{1pt}
\end{example}    





\begin{example}
%\textbf{Example 2:} $L_{r_{12}}=\{t|\mbox{ no }a\mbox{ in } t\mbox{ has an } a \mbox{as child}.\}$\\\\        
                     $L_{r_{6}}=\{t|\mbox{ no }a\mbox{ in } t\mbox{ has an } a \mbox{ as child}.\}$\\\\ 

$\Gamma _{r_{6}} = (\Sigma ,\mathcal{A},\mathcal{A} ',R)$ where,\\

    $\Sigma =\{a_2,b_2,c_0\}$,  
$\mathcal{A} = \{A_1\}$, $F_1=[-1,-1,-1]$
$\mathcal{A}'= \{A_1\}$\\
{\small $A_1=
    \begin{Bmatrix}
        \pstree[nodesep=1pt,levelsep=4ex]{\Tr{c}}
        {}
        ,
        \pstree[nodesep=1pt,levelsep=4ex]{\Tr{a}}
        {
            \Tr{c}
            \Tr{c}
        }
        ,
        \pstree[nodesep=1pt,levelsep=4ex]{\Tr{a}}
        {
            \Tr{b}
            \Tr{b}
        }
        , 
        \pstree[nodesep=1pt,levelsep=4ex]{\Tr{a}}
        {
            \Tr{b}
            \Tr{c}
        }
        ,       
        \pstree[nodesep=1pt,levelsep=4ex]{\Tr{a}}
        {
            \Tr{c}
            \Tr{b}
        }
        ,
        \pstree[nodesep=1pt,levelsep=4ex]{\Tr{b}}
        {
            \Tr{c}
            \Tr{c}
        }
        ,
        \pstree[nodesep=1pt,levelsep=4ex]{\Tr{b}}
        {
            \Tr{a}
            \Tr{a}
        }
        ,
        \pstree[nodesep=1pt,levelsep=4ex]{\Tr{b}}
        {
            \Tr{b}
            \Tr{b}
        }
        ,\\ 
        \pstree[nodesep=1pt,levelsep=4ex]{\Tr{b}}
        {
            \Tr{a}
            \Tr{c}
        }
        ,
        \pstree[nodesep=1pt,levelsep=4ex]{\Tr{b}}
        {
            \Tr{c}
            \Tr{a}
        }       ,
        \pstree[nodesep=1pt,levelsep=4ex]{\Tr{b}}
        {
            \Tr{b}
            \Tr{c}
        }
        ,       
        \pstree[nodesep=1pt,levelsep=4ex]{\Tr{b}}
        {
            \Tr{c}
            \Tr{b}
        }
        ,
        \pstree[nodesep=1pt,levelsep=4ex]{\Tr{b}}
        {
            \Tr{a}
            \Tr{b}
        }
        ,
        \pstree[nodesep=1pt,levelsep=4ex]{\Tr{b}}
        {
            \Tr{b}
            \Tr{a}
        }
     \end{Bmatrix}$}\\\\

$R$ contains rules,\\
   $r_1=(\chi _1,C_1,C_2)$,$r_2=(\chi _2, C_3,C_4)$,  where \\
   $\chi _1=(a,-,-),\chi _2=(b,-,-)$\\        
    $C_1 =(A_1,V_1,1),C_2=(A_1,V_1,2),C_3=(A_1,V_2,1),C_4=(A_1,V_2,2)$\\
        $V_1\in \{c,b\},V_2\in \{a,b,c\}$  \\\\
%    $C_1 =(A_1,c|b,1),C_2=(A_1,c|b,2),C_3=(A_1,a|b|c,1),C_4=(A_1,a|b|c,2)$\\\\      

The production rules are
\begin{eqnarray*}   
P=\{
S &\rightarrow & a(C,C)|a(B,B)|a(B,C)|a(C,B)|b(C,C)|b(A,A)|b(B,B)|b(A,C)|b(A,A)|b(B,A)|b(A,B)|b(C,B)|b(B,C)|C\\
A &\rightarrow & a(C,C)|a(C,B)|a(B,C)|a(B,B)\\
B &\rightarrow & b(A,B)|b(A,C)|b(C,C)|b(C,A)|b(C,B)|b(B,C)|b(B,A)|b(A,A)|b(B,B)\\
C &\rightarrow & c\}
\end{eqnarray*}
  The regular tree grammar corresponding to $\Gamma _{r_{3}}$ is 
  $G_{r_3}=(\{S,A,B,C\},\{a,b,c\},S,P)$\\
\noindent \rule{\textwidth}{1pt}
\end{example} 
\begin{example}
%\textbf{Example 3:} $L_{r_{11}}=\{t|n_a(t)\%2\neq 0\}$\\\\        
                     $L_{r_{4}}=\{t|n_a(t)\%2\neq 0\}$\\\\
$\Gamma _{r_{4}} = (\Sigma ,\mathcal{A},\mathcal{A} ',R)$ where,\\
$\Sigma =\{a_2,b_2,c_0\}$,  
$\mathcal{A} = \{A_1,A_2\}$\\$F_1=[-1,-1,-1],F_2=[0,1,0]$,
$\mathcal{A}'= \{A_1,A_2\}$, where\\
{\small $A_1=
\begin{Bmatrix}
    %   \pstree[nodesep=1pt,levelsep=4ex]{\Tr{b}}
     %   {
      %      \Tr{c}
       %     \Tr{c}
    %    }
     %   ,
        \pstree[nodesep=1pt,levelsep=4ex]{\Tr{b}}
        {
            \Tr{b}
            \Tr{b}
        }
        , 
        \pstree[nodesep=1pt,levelsep=4ex]{\Tr{b}}
        {
            \Tr{b}
            \Tr{c}
        }
        ,       
        \pstree[nodesep=1pt,levelsep=4ex]{\Tr{b}}
        {
            \Tr{c}
            \Tr{b}
        }
        ,
	 \pstree[nodesep=1pt,levelsep=4ex]{\Tr{b}}
        {}
	,
        \pstree[nodesep=1pt,levelsep=4ex]{\Tr{c}}
        {}
\end{Bmatrix}$}\\
{\small $A_2=
    \begin{Bmatrix}
        \pstree[nodesep=1pt,levelsep=4ex]{\Tr{a}}
        {
            \Tr{c}
            \Tr{c}
        }
        ,
        \pstree[nodesep=1pt,levelsep=4ex]{\Tr{a}}
        {
            \Tr{b}
            \Tr{b}
        }
        , 
        \pstree[nodesep=1pt,levelsep=4ex]{\Tr{a}}
        {
            \Tr{b}
            \Tr{c}
        }
        ,       
        \pstree[nodesep=1pt,levelsep=4ex]{\Tr{a}}
        {
            \Tr{c}
            \Tr{b}
        }
        ,
        \pstree[nodesep=1pt,levelsep=4ex]{\Tr{a}}
        {
            \Tr{a}
            \Tr{a}
        }
        ,
        \pstree[nodesep=1pt,levelsep=4ex]{\Tr{b}}
        {
            \Tr{a}
            \Tr{c}
        }
        ,
        \pstree[nodesep=1pt,levelsep=4ex]{\Tr{b}}
        {
            \Tr{b}
            \Tr{a}
        }
        ,       
        \pstree[nodesep=1pt,levelsep=4ex]{\Tr{b}}
        {
            \Tr{a}
            \Tr{b}
        }
        , 
        \pstree[nodesep=1pt,levelsep=4ex]{\Tr{b}}
        {
            \Tr{c}
            \Tr{a}
        }
        ,
        \pstree[nodesep=1pt,levelsep=4ex]{\Tr{a}}
        {}
\end{Bmatrix}$}\\
\\\\
$R$ contains rules,\\
 $r_1=(\chi _1,U_1,U_2))$, $r_2=(\chi _2, U_1,U_2)$, where\\
%  $r_1=(\chi _1,(C_1|C_3),(C_2|C_4))$\\ $r_2=(\chi _2, (C_1|C_3),(C_2|C_4))$\\
  %$r_3=(\chi _3,(C_1|C_4))$\\ $r_4=(\chi _4,(C_1|C_3))$\\  
  %$r_5=(\chi _5,(C_2|C_4))$\\ $r_6=(\chi _6, (C_1|C_3))$\\
  %$r_7=(\chi _7,(C_2|C_4))$\\ $r_8=(\chi _8, (C_1|C_3))$ \\
  %$r_9=(\chi _9,(C_2|C_4))$\\ $r_{10}=(\chi _{10}, (C_1|C_3))$,   where \\
   $\chi _1=(a,-,-),\chi _2=(b,-,-)$,
   $U_1\in\{C_1,C_3\},U_2\in\{C_2,C_4\}$\\
   %\\,\chi _3=(b,a,-),\chi _4=(b,-,a)$, $\chi _5=(a,b,-)$\\
  % $\chi _6=(a,-,b),\chi _7=(a,c,-),\chi _8=(a,-,c),\chi _9=(b,c,-)$, $\chi _{10}=(b,-,c)$\\
    $C_1 =(A_1,V_1,1),C_2=(A_1,V_1,2),C_3=(A_2,V_2,1),C_4=(A_2,V_2,2)$\\
     $V_1\in\{c,b\},V_2\in\{a,b\}$\\

  The production rules are
\begin{eqnarray*}   
P=\{
S &\rightarrow & a(C,C)|a(B,B)|a(B,C)|a(C,B)|a(A,A)|b(A,C)|b(B,A)|b(A,B)|b(C,A)|A\\
      & &b(S,B)|b(B,S)|b(S,C)|b(C,S)\\        
A &\rightarrow & a(C,C)|a(C,B)|a(B,C)|a(B,B)|a(C,A')|a(C,B')|a(B,A')|a(B,B')|\\& &
                  a(A',C)|a(A',B)|a(B',C)|a(B',B)|a(A',A')|a(A',B')|a(B',A')|a(B',B')\\
                  & & a(A,A)|a(B,B)\\
B &\rightarrow & b(C,C)|b(C,B)|b(B,C)|b(B,B)|b(C,A')|b(C,B')|b(B,A')|b(B,B')|\\& &
                 b(A',C)|b(A',B)|b(B',C)|b(B',B)|b(A',A')|b(A',B')|b(B',A')|b(B',B')\\
                & & b(A,A)|b(B,B)\\
%A' &\rightarrow & a(A,C)|a(A,B)|a(C,A)|a(B,A)|a(A,A')|a(A',A)\\  
A' &\rightarrow & a(S,C)|a(S,B)|a(C,S)|a(B,S)|a(S,A')|a(A',S)|a(S,B')|a(B',S)\\ 
B' &\rightarrow & b(A',C)|b(A',B)|b(C,A')|b(B,A')|b(A',A')|b(A,A)|b(B',A')|b(A',B')\\
C &\rightarrow & c\}
\end{eqnarray*}
  The regular tree grammar corresponding to $\Gamma _{r_{4}}$ is 
  $G_{r_4}=(\{S,A,A',B,B',C\},\{a,b,c\},S,P)$

\noindent \rule{\textwidth}{1pt}
\end{example}    
   
\emph{\textbf{Result 1:} Given a tree insertion system, we can construct an equivalent regular tree grammar.}\\
\noindent \rule{\textwidth}{1pt}

\subsection{Finite state Bottom-up Tree Automata to Insertion System}
Since there exists a bottom-up finite tree automata for accepting a regular tree language,it is enough to simulate that 
automata using the tree insertion system, to show the equivalence of tree insertion system and regular tree grammars.

\subsubsection{Method of conversion}
For a given deterministic bottom-up tree automata $D=(\Sigma ' , Q, Q_f, \Delta)$ we can construct a 
tree insertion system $\Gamma =(\Sigma,\mathcal{A} ,\mathcal{A} ',R)$, where
\begin{itemize}
	\item $\Sigma = \Sigma '$
	\item If transitions are non-recursive

		For each transition of the form $p(q_1,q_2,\cdots ,q_k)\rightarrow q_g$, where $q_1,q_2,\cdots ,q_k,q_g\in Q$, $p\in \Sigma '$, $k$ 
		is the arity of node with label $p$ and $q_g\in Q_f$, 
		{\tiny
		$
		\pstree[nodesep=1pt,levelsep=4ex]{\Tr{p}}
        	{
            		\Tr{q_1}
            		\Tr{q_2}
			\Tr{\cdots }
			\Tr{q_i}
			\Tr{\cdots }
			\Tr{q_k}
	        }
		$} will be in $ \mathcal{A '}$

	$\forall q_i,1\leq i\leq k$, %if $c_1(q_{i_1})\rightarrow c_2(q_{i_2})\rightarrow \cdots \rightarrow c_n(q_{i_n})\rightarrow q_i$
	if $\{c_1(q_{i_1})\rightarrow q_i,c_2(q_{i_2})\rightarrow q_{i_1}\cdots c_n(q_{i_n})\rightarrow q_{i_{n-1}}\}\in \Delta$ where $q_{i_n}=q_i$
	, $c_i\in \Sigma ', 1\leq i\leq n$ and $\forall i,j,q_{i_j}\in Q,1\leq j\leq n$,
		{\tiny
		$	
		\begin{comment}
		\pstree[nodesep=1pt,levelsep=4ex]{\Tr{p}}
        	{
            		\Tr{q_1}
            		\Tr{q_2}
			\Tr{\cdots }
			\Tr{q_i}
			\Tr{\cdots }
			\Tr{q_f}
	        }
		\Rightarrow
		\end{comment}
		\pstree[nodesep=1pt,levelsep=4ex]{\Tr{p}}
        	{
            		\Tr{q_1}
            		\Tr{q_2}
			\Tr{\cdots }
			\pstree[nodesep=1pt,levelsep=1ex]{\Tr{c_1}}
			{
%				\pstree[nodesep=1pt,levelsep=1ex]{\Tr{c_2}}
				{
					\pstree[nodesep=1pt,levelsep=1ex]{\Tr{.}}
					{
						\pstree[nodesep=1pt,levelsep=1ex]{\Tr{.}}
						{
							\pstree[nodesep=1pt,levelsep=1ex]{\Tr{.}}
							{
		`						\Tr{c_n}
							}
						}
					}
				}
			}
			\Tr{\cdots }
			\Tr{q_k}
	        }
		$
		}
		will be in $\mathcal{A '}$

	If $\{c_1(q_{i_1})\rightarrow q_i,c_2(q_{i_2})\rightarrow q_{i_1}\cdots c_n(q_{i_n})\rightarrow q_{i_{n-1}}, a\rightarrow q_{i_n}\}\in \Delta$ 
	where  $a,c_i\in \Sigma ', 1\leq i\leq n$ and \hbox{$\forall i,j,q_{i_j}\in Q,1\leq j\leq n$,}
		{\tiny
		$	
		\begin{comment}
		\pstree[nodesep=1pt,levelsep=4ex]{\Tr{p}}
        	{
            		\Tr{q_1}
            		\Tr{q_2}
			\Tr{\cdots }
			\Tr{q_i}
			\Tr{\cdots }
			\Tr{q_f}
	        }
		\Rightarrow
		\end{comment}
		\pstree[nodesep=1pt,levelsep=4ex]{\Tr{p}}
        	{
            		\Tr{q_1}
            		\Tr{q_2}
			\Tr{\cdots }
			\pstree[nodesep=1pt,levelsep=1ex]{\Tr{c_1}}
			{
%				\pstree[nodesep=1pt,levelsep=1ex]{\Tr{c_2}}
				{
					\pstree[nodesep=1pt,levelsep=1ex]{\Tr{.}}
					{
						\pstree[nodesep=1pt,levelsep=1ex]{\Tr{.}}
						{
							\pstree[nodesep=1pt,levelsep=1ex]{\Tr{.}}
							{
								\pstree[nodesep=1pt,levelsep=2ex]{\Tr{c_n}}
								{
		`							\Tr{a}
								}
							}
						}
					}
				}
			}
			\Tr{\cdots }
			\Tr{q_k}
	        }
		$
		}
		will be in $ \mathcal{A '}$

	For each transition of the form $p(q_1,q_2,\cdots ,q_k)\rightarrow q_g$, where $q_1,q_2,\cdots ,q_k,q_g\in Q$,  $p\in \Sigma '$, $k$ 
	is the arity of node with label $p$ and $q_g\in Q_f$ 
	$\forall q_i,1\leq i\leq p$ 
	if $\{c_1(q_{i_1})\rightarrow q_i,c_2(q_{i_2})\rightarrow q_{i_1}\cdots c_n(q_{i_n})\rightarrow q_{i_{n-1}}\}\in \Delta$ where $q_{i_n}=q_i$
	, $c_i\in \Sigma ', 1\leq i\leq n$ and $\forall i,j,q_{i_j}\in Q,1\leq j\leq n$,
		{\small
		$
		\pstree[nodesep=1pt,levelsep=1ex]{\Tr{c_1}}
		{
	%		\pstree[nodesep=1pt,levelsep=1ex]{\Tr{c_2}}
			{
			%	\pstree[nodesep=1pt,levelsep=1ex]{\Tr{.}}
				{
					\pstree[nodesep=1pt,levelsep=1ex]{\Tr{.}}
					{
						\pstree[nodesep=1pt,levelsep=1ex]{\Tr{.}}
						{
							\Tr{c_n}
						}
					}
				}
			}
		}
		$} will be in $ \mathcal{A}-\mathcal{A}'$

	If $\{c_1(q_{i_1})\rightarrow q_i,c_2(q_{i_2})\rightarrow q_{i_1}\cdots c_n(q_{i_n})\rightarrow q_{i_{n-1}}, a\rightarrow q_{i_n}\}\in \Delta$ 
	where  $a,c_i\in \Sigma ', 1\leq i\leq n$ and \hbox{$\forall i,j,q_{i_j}\in Q,1\leq j\leq n$,}
		{\small
		$
		\pstree[nodesep=1pt,levelsep=1ex]{\Tr{c_1}}
		{
		%	\pstree[nodesep=1pt,levelsep=1ex]{\Tr{c_2}}
			{
		%		\pstree[nodesep=1pt,levelsep=1ex]{\Tr{.}}
				{
					\pstree[nodesep=1pt,levelsep=1ex]{\Tr{.}}
					{
						\pstree[nodesep=1pt,levelsep=1ex]{\Tr{.}}
						{
							\pstree[nodesep=1pt,levelsep=2ex]{\Tr{c_n}}
							{
	`							\Tr{a}
							}
						}
					}
				}
			}
		}
		$} will be in $ \mathcal{A}-\mathcal{A}'$

		Set flag $F_i=[x_i,y_i,z_i]$ associated with each $A_i\in\mathcal{A}$ as  [-1,-1,-1].
	\item If transitions are recursive

		Let $AF$ and $AN$ are two temporary axiom sets
		
		For each transition of the form $p(q_1,q_2,\cdots ,q_k)\rightarrow q_g$, where  $q_1,q_2,\cdots ,q_k,q_g\in Q$, $p\in \Sigma '$, 
		$k$ is the arity of node with label $p$,
		{\tiny
		$
		\pstree[nodesep=1pt,levelsep=4ex]{\Tr{p}}
        	{
            		\Tr{q_1}
            		\Tr{q_2}
%			\Tr{\cdots }
%			\Tr{q_i}
			\Tr{\cdots }
			\Tr{q_k}
	        }
		$} will be in $AF$ if  $q_g\in Q_f$ and it will be in $AN$ if $q_g\notin Q_f$.
\begin{comment}

		For each transition of the form $p(q_1,q_2,\cdots ,q_k)\rightarrow q_g$, where  $q_1,q_2,\cdots ,q_g\in Q$, $p\in \Sigma '$, 
		$k$ is the arity of node with label $p$ and $q_g\notin Q_f$ 
		{\small
		$
		\pstree[nodesep=1pt,levelsep=4ex]{\Tr{p}}
        	{
            		\Tr{q_1}
            		\Tr{q_2}
%			\Tr{\cdots }
%			\Tr{q_i}
			\Tr{\cdots }
			\Tr{q_k}
	        }
		$}$\in AN $
\end{comment}		
		
		Let $t=$
%		$\forall t\in AF$, if $t$ is of the form 
		{\tiny
		$
		\pstree[nodesep=1pt,levelsep=4ex]{\Tr{p}}
        	{
            		\Tr{q_1}
            		\Tr{q_2}
%			\Tr{\cdots }
%			\Tr{q_i}
			\Tr{\cdots }
			\Tr{q_k}
	        }
		$}%with all  $q_1,q_2,\cdots ,q_k\in Q_f$, then $p\in AF$
		and $t'=$
		{\tiny
		$
		\pstree[nodesep=1pt,levelsep=4ex]{\Tr{p}}
        	{
            		\Tr{t_1}
            		\Tr{t_2}
%			\Tr{\cdots }
%			\Tr{t_i}
			\Tr{\cdots }
			\Tr{t_k}
	        }
		$}%$\in AF$	 

	$\forall t\in AF$, if  all  $q_1,q_2,\cdots ,q_k\in Q_f$, then $p$ will be in $AF$

	$\forall t\in AN$, if  all  $q_1,q_2,\cdots ,q_k\notin Q_f$, then $p$ will be in $AN$

	$\forall t\in AF$,  $t'$ will be in $AF$ with $t_i$ having the label of root node of some $t''\in AF$, $\forall q_i\in Q_f$. 

	$\forall t\in AN$,  $t'$ will be in $AN$ with $t_i$ having the label of root node of some $t''\in AN$, $\forall q_i\notin Q_f$. 

	$\forall t\in AF$,  $t'$ will be in $AF$ with $t_i$ having the label of root node of some $t''\in AN$, $\forall q_i\notin Q_f$. 

	$\forall t\in AN$,  $t'$ will be in $AN$ with $t_i$ having the label of root node of some $t''\in AF$, $\forall q_i\in Q_f$.


 	Set flag $F_{AN}=[x,y,z]$ associated with $AN$ as  [-1,-1,-1].

 Set  flag $F_{AF}=[x,y,z]$ associated with an axiom $AF$, which has a constraint on 
$\sigma_i\in\Sigma$, as follows.
\begin{itemize}
	\item $x=0$.
	\item If $\exists t=
		{\tiny
		\pstree[nodesep=1pt,levelsep=4ex]{\Tr{p}}
        	{
            		\Tr{p_1}
            		\Tr{p_2}
%			\Tr{\cdots }
%			\Tr{q_i}
			\Tr{\cdots }
			\Tr{p_k}
	        }}\in AF$ , $k\geq 0$, if  $p,p_1,p_2,\cdots ,p_k\neq \sigma_i$ then $y_i=0$.
	\item Else if $\exists t\in AF$, with $n$ nodes of $t$ have label $\sigma_i$, then $y_i=z_i=n$.
	\item If $t'=
		{\tiny
		\pstree[nodesep=1pt,levelsep=4ex]{\Tr{q}}
        	{
            		\Tr{q_1}
            		\Tr{q_2}
%			\Tr{\cdots }
%			\Tr{q_i}
			\Tr{\cdots }
			\Tr{q_k'}
	        }}\in AF$ , $k'\geq 0$ having $m\geq n$ nodes of $t$ are with label $\sigma_i$.
		\begin{itemize}
			\item $z=(m-n)-1$, if $y=0$.
			\item $z=(m-1)$, if $y\neq 0$.
		\end{itemize}

	\end{itemize}
	
		Now $\mathcal{A} '=AF$ and $\mathcal{A}=AF\cup AN$


	\item For each transitions of the form $a\rightarrow q_a \in \Delta$ where, $a\in \Sigma ', q_a\in Q$, $a \mbox{ will be in } \mathcal{A} ', 
		\mbox{ if }$\hbox{$ q_a\in Q_f$}$\mbox{ and } a $ will be in $\mathcal{A-A}',\mbox{ if } q_a\notin Q_f $ 
	\item $\forall t\in \mathcal{A}$ which is of the form 
		{\tiny
		$
		\pstree[nodesep=1pt,levelsep=4ex]{\Tr{p}}
        	{
		\Tr{t_1}
            		\Tr{t_2}
%			\Tr{\cdots }
%			\Tr{t_i}
			\Tr{\cdots }
			\Tr{t_k}
	        }
		$}	

		$r=(\chi,C_1,C_2,\cdots ,C_k)\in R$ where, $\chi=(p,-,\cdots,-)$ and $\forall i\leq k, C_i=(X,a,i)$ where, \hbox{$X\in \mathcal{A},$} 
		$\mbox{ for some }$\hbox{$ t'\in X,$}$ t'$ has $a$ as root.

		And $\forall t_i, 1\leq i\leq k$ which is of the form
		{\tiny
		$
		\pstree[nodesep=1pt,levelsep=4ex]{\Tr{p'}}
        	{
            		\Tr{t_1}
            		\Tr{t_2}
%			\Tr{\cdots }
%			\Tr{t_i}
			\Tr{\cdots }
			\Tr{t_{k'}}
	        }
		$}	
	
		$r=(\chi,C_1,C_2,\cdots ,C_{k'})\in R$ where, $\chi=(p',-,\cdots,-)$ and $\forall i\leq k', C_i=(X,a,i)$ where, \hbox{$X\in \mathcal{A},$} 
		$\mbox{ for some } t'\in X,t'$ has $a$ as root.



\end{itemize}

\begin{example}
    %\textbf{Example 3:} $L_{r_{13}}=\{a(b^i(g),c^j(g)),i,j\ge 1\mbox{, } i\%3 = 0\mbox{ and }j\%2 =0\}$
                         $L_{r_{1}}=\{a(b^i(h),c^j(h)),i,j\ge 1\}$   
Consider the tree automata $D_{r_{1}}=(Q,\Sigma ',Q_f, \Delta)$ which accepts the language $L_{r_{1}}$ over the 
alphabet $\Sigma ' =\{a_2,b_1,c_1,h_0\}$ 
where $Q=\{q_a,q_b,q_c,q_h\},Q_f=\{q_a\}$ and $\Delta$ contains following transitions.\\

      \begin{tabular}{llllll}
$h\rightarrow q_h$ & $ b(q_h) \rightarrow q_b$ & $b(q_b) \rightarrow q_b$ & 
 $ c(q_h) \rightarrow q_c$ & $ c(q_c) \rightarrow q_c$ & $ a(q_b,q_c) \rightarrow q_a$\\
\end{tabular}
%\\Create initial axiom $A_1$ from transitions which directly leads to final state $q_a$. 
%Since we cannot generate the axioms directly, do it as s sequence of steps.

{\small 
%    \begin{eqnarray*}
$
    A_1=
     \begin{Bmatrix}
        \pstree[nodesep=1pt,levelsep=3ex]{\Tr{a}}
        {
                \Tr{b}
            	\Tr{c}
        }
        ,
        \pstree[nodesep=1pt,levelsep=3ex]{\Tr{a}}
        {
            \pstree[nodesep=1pt,levelsep=3ex]{\Tr{b}}
            {
                \Tr{h}
            	        
            }
            \pstree[nodesep=1pt,levelsep=3ex]{\Tr{c}}
            {
                \Tr{h}
            }
        }
     \end{Bmatrix}
 %    \end{eqnarray*}
$
}, 
%Generate the axioms for states $q_b$ and $q_c$ which are the arguments of transition that leads to final state and for $q_g$ which is an 
%independent transition.
{\small
%    \begin{eqnarray*}
$
A_2 =
    \begin{Bmatrix}
     \pstree[nodesep=1pt,levelsep=3ex]{\Tr{c}}
     {
     }
     \hspace{2mm}
     ,
     \hspace{2mm}
     \pstree[nodesep=1pt,levelsep=3ex]{\Tr{c}}
     {         
            \Tr{h}
     }
    \hspace{2mm}
     ,
     \hspace{2mm}

     \pstree[nodesep=1pt,levelsep=3ex]{\Tr{b}}
     {
     }
     \hspace{2mm}
     ,
     \hspace{2mm}
     \pstree[nodesep=1pt,levelsep=3ex]{\Tr{b}}
     {
                \Tr{h}
     }
     \hspace{2mm}
     ,
     \hspace{2mm}

     \pstree[nodesep=1pt,levelsep=3ex]{\Tr{h}}{}
    \end{Bmatrix}
%\end{eqnarray*}
$
}\\
%Since we could generate all trees in the axioms directly from the transitions itself, no need to set the tuple associated with the axioms.
%Create rules for generating all trees both in $A_1$ and $A_2$.\\\\
$R$ contains rules,\\
%   $r_1=(\chi _1,C_1,C_2)$, $r_2=(\chi _2, C_1|C_4)$,  $r_3=(\chi _3, C_3|C_4)$ where \\
   $r_1=(\chi _1,C_1,C_2)$, $r_2=(\chi _2, U_1)$,  $r_3=(\chi _3, U_2)$ where \\
   $\chi _1=(a,-,-),\chi _2=(b,-),\chi_3 =(c,-)$, $U_1\in \{C_1,C_4\},U_2\in \{C_3,C_4\}$\\
    $C_1 =(A_2,b,1),C_2=(A_2,c,2),C_3=(A_2,c,1),C_4=(A_2,h,1)$\\\\
So the tree insertion system equivalent to $D_{r_{1}}$ is            
$\Gamma _{r_{1}} = (\Sigma ,\mathcal{A},\mathcal{A} ',R)$ where,\\\\
    $\Sigma =\{a_2,b_1,c_1,h_0\}$, $\mathcal{A} = \{A_1,A_2\}$,$F_1=[-1,-1,-1],F_2=[-1,-1,-1]$,$\mathcal{A}'= \{A_1\}$\\
\noindent \rule{\textwidth}{1pt}
\end{example}    


\begin{example}
    %\textbf{Example 3:} $L_{r_{13}}=\{a(b^i(g),c^j(g)),i,j\ge 1\mbox{, } i\%3 = 0\mbox{ and }j\%2 =0\}$
                         $L_{r_{5}}=\{a(b^i(h),c^j(h)),i,j\ge 1\mbox{, } i\%3 = 0\mbox{ and }j\%2 =0\}$   
Consider the tree automata $D_{r_{5}}=(Q,\Sigma ',Q_f, \Delta)$ which accepts the language $L_{r_{5}}$ over the 
alphabet $\Sigma ' =\{a_2,b_1,c_1,h_0\}$ 
where $Q=\{q_a,q_b,q_{b_1},q_{b_2},q_c,q_{c_1},q_h\},Q_f=\{q_a\}$ and $\Delta$ contains following transitions.\\

      \begin{tabular}{llllll}
$h\rightarrow q_h$ & $ b(q_h) \rightarrow q_{b_1}$ & $ b(q_{b_1}) \rightarrow q_{b_2}$ & $b(q_{b_2}) \rightarrow q_b$ & 
  $ b(q_b) \rightarrow q_{b_1}$\\ 
& $ c(q_h) \rightarrow q_{c_1}$ & $ c(q_{c_1}) \rightarrow q_c$ & $ c(q_c) \rightarrow q_{c_1}$ & $ a(q_b,q_c) \rightarrow q_a$\\
\end{tabular}
%\\Create initial axiom $A_1$ from transitions which directly leads to final state $q_a$.
%Since we cannot generate the axioms directly, do it as s sequence of steps.

{\small 
%    \begin{eqnarray*}
$
    A_1=
 \begin{comment}
   \begin{Bmatrix}
        \pstree[nodesep=1pt,levelsep=3ex]{\Tr{a}}
        {
            \Tr{q_b}
            \Tr{q_c}
        }
    \end{Bmatrix}   
        \Rightarrow \cdots \Rightarrow
    \begin{Bmatrix}
        \pstree[nodesep=1pt,levelsep=3ex]{\Tr{a}}
        {
            \pstree[nodesep=1pt,levelsep=3ex]{\Tr{b}}
            {
                 \Tr{q_{b_2}}
            }
            \pstree[nodesep=1pt,levelsep=3ex]{\Tr{c}}
            {
                 \Tr{q_{c_1}}
            }
        }
     \end{Bmatrix}
        \Rightarrow   
     \begin{Bmatrix}
        \pstree[nodesep=1pt,levelsep=3ex]{\Tr{a}}
        {
            \pstree[nodesep=1pt,levelsep=3ex]{\Tr{b}}
            {
                \pstree[nodesep=1pt,levelsep=3ex]{\Tr{b}}
                {
                    \Tr{q_{b_1}}
                }
            }
            \pstree[nodesep=1pt,levelsep=3ex]{\Tr{c}}
            {
                \pstree[nodesep=1pt,levelsep=3ex]{\Tr{c}}
                {
                    \pstree[nodesep=1pt,levelsep=2ex]{\Tr{q_{c}}}
                    {
                        {\tiny{{\psset{linestyle=none}\TR{(loop)}}}}
                    }
                }
            }
        }
  %   \end{Bmatrix}
  %      or
   %  \begin{Bmatrix}
        ,
        \pstree[nodesep=1pt,levelsep=3ex]{\Tr{a}}
        {
            \pstree[nodesep=1pt,levelsep=3ex]{\Tr{b}}
            {
                \pstree[nodesep=1pt,levelsep=3ex]{\Tr{b}}
                {
                     \Tr{q_{b_1}}
                }
            }
            \pstree[nodesep=1pt,levelsep=3ex]{\Tr{c}}
            {
                \pstree[nodesep=1pt,levelsep=3ex]{\Tr{c}}
                {
                    \Tr{q_{h}}
                }
            }
        }
     \end{Bmatrix}\\
         &\Rightarrow &
     \begin{Bmatrix}
        \pstree[nodesep=1pt,levelsep=3ex]{\Tr{a}}
        {
            \pstree[nodesep=1pt,levelsep=3ex]{\Tr{b}}
            {
                \pstree[nodesep=1pt,levelsep=3ex]{\Tr{b}}
                {
                    \pstree[nodesep=1pt,levelsep=3ex]{\Tr{b}}
                    {
                        \pstree[nodesep=1pt,levelsep=2ex]{\Tr{q_{b}}}
                        {
                            {\tiny{\psset{linestyle=none}\TR{(loop)}}}
                        }
                    }
                 }
            }
            \pstree[nodesep=1pt,levelsep=3ex]{\Tr{c}}
            {
                \pstree[nodesep=1pt,levelsep=3ex]{\Tr{c}}
                {
                    \pstree[nodesep=1pt,levelsep=2ex]{\Tr{q_{c}}}
                    {
                        {\tiny{\psset{linestyle=none}\TR{(loop)}}}
                    }
                }
            }
        }
    % \end{Bmatrix}
    %    or
    % \begin{Bmatrix}
        ,
        \pstree[nodesep=1pt,levelsep=3ex]{\Tr{a}}
        {
            \pstree[nodesep=1pt,levelsep=3ex]{\Tr{b}}
            {
                \pstree[nodesep=1pt,levelsep=3ex]{\Tr{b}}
                {
                    \pstree[nodesep=1pt,levelsep=3ex]{\Tr{b}}
                    {
                        \Tr{q_h}
                    }
                }
            }
            \pstree[nodesep=1pt,levelsep=3ex]{\Tr{c}}
            {
                \pstree[nodesep=1pt,levelsep=3ex]{\Tr{c}}
                {
                    \Tr{q_{h}}
                }
            }
        }
     \end{Bmatrix}
          \Rightarrow   
     \begin{Bmatrix}
        \pstree[nodesep=1pt,levelsep=3ex]{\Tr{a}}
        {
            \pstree[nodesep=1pt,levelsep=3ex]{\Tr{b}}
            {
                \pstree[nodesep=1pt,levelsep=3ex]{\Tr{b}}
                {
                    \Tr{b}
                 }
            }
            \pstree[nodesep=1pt,levelsep=3ex]{\Tr{c}}
            {
                \pstree[nodesep=1pt,levelsep=3ex]{\Tr{c}}
                {
                }
            }
        }
    % \end{Bmatrix}
    %    or
    % \begin{Bmatrix}
        ,
        \pstree[nodesep=1pt,levelsep=3ex]{\Tr{a}}
        {
            \pstree[nodesep=1pt,levelsep=3ex]{\Tr{b}}
            {
                \pstree[nodesep=1pt,levelsep=3ex]{\Tr{b}}
                {
                    \pstree[nodesep=1pt,levelsep=3ex]{\Tr{b}}
                    {
                        \Tr{q_h}
                    }
                }
            }
            \pstree[nodesep=1pt,levelsep=3ex]{\Tr{c}}
            {
                \pstree[nodesep=1pt,levelsep=3ex]{\Tr{c}}
                {
                    \Tr{q_{h}}
                }
            }
        }
     \end{Bmatrix}
        \Rightarrow 
\end{comment}
     \begin{Bmatrix}
        \pstree[nodesep=1pt,levelsep=3ex]{\Tr{a}}
        {
            \pstree[nodesep=1pt,levelsep=3ex]{\Tr{b}}
            {
                \pstree[nodesep=1pt,levelsep=3ex]{\Tr{b}}
                {
                    \Tr{b}
                 }
            }
            \pstree[nodesep=1pt,levelsep=3ex]{\Tr{c}}
            {
                \pstree[nodesep=1pt,levelsep=3ex]{\Tr{c}}
                {
                }
            }
        }
        ,
        \pstree[nodesep=1pt,levelsep=3ex]{\Tr{a}}
        {
            \pstree[nodesep=1pt,levelsep=3ex]{\Tr{b}}
            {
                \pstree[nodesep=1pt,levelsep=3ex]{\Tr{b}}
                {
                    \pstree[nodesep=1pt,levelsep=3ex]{\Tr{b}}
                    {
                        \Tr{h}
                    }
                }
            }
            \pstree[nodesep=1pt,levelsep=3ex]{\Tr{c}}
            {
                \pstree[nodesep=1pt,levelsep=3ex]{\Tr{c}}
                {
                    \Tr{h}
                }
            }
        }
     \end{Bmatrix}
 %    \end{eqnarray*}
$
  }, 
%Generate the axioms for states $q_b$ and $q_c$ which are the arguments of transition that leads to final state and for $q_g$ which is an 
%independent transition.
{\small
%    \begin{eqnarray*}
$
A_2 =
 \begin{comment}
   \begin{Bmatrix}
     \pstree[nodesep=1pt,levelsep=3ex]{\Tr{q_c}}{},
     \pstree[nodesep=1pt,levelsep=3ex]{\Tr{q_b}}{}
    % \pstree[nodesep=1pt,levelsep=3ex]{\Tr{q_h}}{}
    \end{Bmatrix}
        \Rightarrow \cdots \Rightarrow
    \begin{Bmatrix}     
     \pstree[nodesep=1pt,levelsep=3ex]{\Tr{c}}
     {
         \Tr{q_{c_1}}
     }
     ,
     \pstree[nodesep=1pt,levelsep=3ex]{\Tr{b}}
     {
         \Tr{q_{b_2}}
     }
     ,
     \pstree[nodesep=1pt,levelsep=3ex]{\Tr{h}}{}
   \end{Bmatrix}
   \Rightarrow
   \begin{Bmatrix}
     \pstree[nodesep=1pt,levelsep=3ex]{\Tr{c}}
     {
         \pstree[nodesep=1pt,levelsep=3ex]{\Tr{c}}
         {
            \pstree[nodesep=1pt,levelsep=2ex]{\Tr{q_c}}
            {\tiny{\psset{linestyle=none}\TR{(loop)}}}

         }
     }
    % or
        ,
     \pstree[nodesep=1pt,levelsep=3ex]{\Tr{c}}
     {
         \pstree[nodesep=1pt,levelsep=3ex]{\Tr{c}}
         {
            \Tr{q_h}
         }
     }
    ,
     \pstree[nodesep=1pt,levelsep=3ex]{\Tr{b}}
     {
        \pstree[nodesep=1pt,levelsep=3ex]{\Tr{b}}
        {
            \Tr{q_{b_1}}
        }
     }
     ,
     \pstree[nodesep=1pt,levelsep=3ex]{\Tr{h}}{}
    \end{Bmatrix}
%     \Rightarrow 
  \begin{Bmatrix}
     \pstree[nodesep=1pt,levelsep=3ex]{\Tr{c}}
     {
            \Tr{c}
     }
     ,
     \pstree[nodesep=1pt,levelsep=3ex]{\Tr{c}}
     {
         \pstree[nodesep=1pt,levelsep=3ex]{\Tr{c}}
         {
            \Tr{h}
         }
     }
    ,
     \pstree[nodesep=1pt,levelsep=3ex]{\Tr{b}}
     {
        \pstree[nodesep=1pt,levelsep=3ex]{\Tr{b}}
        {
            \pstree[nodesep=1pt,levelsep=3ex]{\Tr{b}}
            {
                \pstree[nodesep=1pt,levelsep=2ex]{\Tr{q_b}}
                {
                    {\tiny{\psset{linestyle=none}\TR{(loop)}} }
                } 
            }
        }
     }
    % or
        ,
     \pstree[nodesep=1pt,levelsep=3ex]{\Tr{b}}
     {
        \pstree[nodesep=1pt,levelsep=3ex]{\Tr{b}}
        {
            \pstree[nodesep=1pt,levelsep=3ex]{\Tr{b}}
            {
                \Tr{q_h}
            }
        }
     }
     ,
     \pstree[nodesep=1pt,levelsep=3ex]{\Tr{h}}{}
    \end{Bmatrix}
        \Rightarrow
\end{comment}
   \begin{Bmatrix}
     \pstree[nodesep=1pt,levelsep=3ex]{\Tr{c}}
     {
            \Tr{c}
     }
     \hspace{2mm}
     ,
     \hspace{2mm}
     \pstree[nodesep=1pt,levelsep=3ex]{\Tr{c}}
     {
         \pstree[nodesep=1pt,levelsep=3ex]{\Tr{c}}
         {
            \Tr{h}
         }
     }
    \hspace{2mm}
     ,
     \hspace{2mm}

     \pstree[nodesep=1pt,levelsep=3ex]{\Tr{b}}
     {
        \pstree[nodesep=1pt,levelsep=3ex]{\Tr{b}}
        {
            \Tr{b}
        }
     }
     \hspace{2mm}
     ,
     \hspace{2mm}
     \pstree[nodesep=1pt,levelsep=3ex]{\Tr{b}}
     {
        \pstree[nodesep=1pt,levelsep=3ex]{\Tr{b}}
        {
            \pstree[nodesep=1pt,levelsep=3ex]{\Tr{b}}
            {
                \Tr{h}
            }
        }
     }
     \hspace{2mm}
     ,
     \hspace{2mm}

     \pstree[nodesep=1pt,levelsep=3ex]{\Tr{h}}{}
    \end{Bmatrix}
%\end{eqnarray*}
$
}\\
%Since we could generate all trees in the axioms directly from the transitions itself, no need to set the tuple associated with the axioms.
%Create rules for generating all trees both in $A_1$ and $A_2$.\\\\
$R$ contains rules,\\
%   $r_1=(\chi _1,C_1,C_2)$, $r_2=(\chi _2, C_1|C_4)$,  $r_3=(\chi _3, C_3|C_4)$ where \\
   $r_1=(\chi _1,C_1,C_2)$, $r_2=(\chi _2, U_1)$,  $r_3=(\chi _3, U_2)$ where \\
   $\chi _1=(a,-,-),\chi _2=(b,-),\chi_3 =(c,-)$, $U_1\in \{C_1,C_4\},U_2\in \{C_3,C_4\}$\\
    $C_1 =(A_2,b,1),C_2=(A_2,c,2),C_3=(A_2,c,1),C_4=(A_2,h,1)$\\\\
So the tree insertion system equivalent to $D_{r_{5}}$ is            
$\Gamma _{r_{5}} = (\Sigma ,\mathcal{A},\mathcal{A} ',R)$ where,\\\\
    $\Sigma =\{a_2,b_1,c_1,h_0\}$, $\mathcal{A} = \{A_1,A_2\}$,$F_1=[-1,-1,-1],F_2=[-1,-1,-1]$,$\mathcal{A}'= \{A_1\}$\\
\noindent \rule{\textwidth}{1pt}
\end{example}    

\begin{example}
%\textbf{Example 1:} $L_{r_{10}}=\{t | t\mbox{ has  }a\mbox{ as root}\}$\\\\
                     $L_{r_{3}}=\{t | t\mbox{ has  }a\mbox{ as root}\}$\\\\
Consider the tree automata $D_{r_{3}}=(Q,\Sigma ' ,Q_f, \Delta)$ which accepts the language $L_{r_{3}}$ over 
the alphabet $\Sigma ' =\{a_2,b_2,c_0\}$ 
where $Q=\{q_a,q_b,q_c\},Q_f=\{q_a\}$ and $\Delta$ contains following transitions.\\

\begin{tabular}{llllll}
$c\rightarrow q_c$ & $ b(q_c,q_c) \rightarrow q_b$ & $ b(q_a,q_a) \rightarrow q_b$ & $b(q_b,q_b) \rightarrow q_b$ & 
$b(q_a,q_c) \rightarrow q_b$\\
$b(q_c,q_a) \rightarrow q_b$ & $ b(q_b,q_c) \rightarrow q_b$ & $ b(q_c,q_b) \rightarrow q_b$ & $b(q_a,q_b) \rightarrow q_b$ & 
$b(q_b,q_a) \rightarrow q_b$\\
$ a(q_c,q_c) \rightarrow q_a$ & $ a(q_a,q_a) \rightarrow q_a$ & $a(q_b,q_b) \rightarrow q_a$ & $a(q_a,q_c) \rightarrow q_a$ &
$ a(q_c,q_a) \rightarrow q_a$\\
$ a(q_b,q_c) \rightarrow q_a$ & $ a(q_c,q_b) \rightarrow q_a$ & $a(q_a,q_b) \rightarrow q_a$ & $a(q_b,q_a) \rightarrow q_a$\\
\end{tabular}
\\Create initial axiom $A_1$ from transitions which directly leads to final state $q_a$.

{\small $A_1=
    \begin{Bmatrix}
        \pstree[nodesep=1pt,levelsep=4ex]{\Tr{a}}
        {
            \Tr{c}
            \Tr{c}
        }
        ,
        \pstree[nodesep=1pt,levelsep=4ex]{\Tr{a}}
        {
            \Tr{a}
            \Tr{a}
        }
        ,
        \pstree[nodesep=1pt,levelsep=4ex]{\Tr{a}}
        {
            \Tr{b}
            \Tr{b}
        }
        , 
         \pstree[nodesep=1pt,levelsep=4ex]{\Tr{a}}
        {
            \Tr{a}
            \Tr{c}
        }
        ,
        \pstree[nodesep=1pt,levelsep=4ex]{\Tr{a}}
        {
            \Tr{c}
            \Tr{a}
        }       ,
        \pstree[nodesep=1pt,levelsep=4ex]{\Tr{a}}
        {
            \Tr{b}
            \Tr{c}
        }
        ,       
        \pstree[nodesep=1pt,levelsep=4ex]{\Tr{a}}
        {
            \Tr{c}
            \Tr{b}
        }
        ,
        \pstree[nodesep=1pt,levelsep=4ex]{\Tr{a}}
        {
            \Tr{a}
            \Tr{b}
        }
        ,
        \pstree[nodesep=1pt,levelsep=4ex]{\Tr{a}}
        {
            \Tr{b}
            \Tr{a}
        }
   \end{Bmatrix}$}\\
Create other axiom $A_2$ from transitions that leads to states which are the arguments of final state transitions.

{\small
    $A_2=
\begin{Bmatrix}
       \pstree[nodesep=1pt,levelsep=4ex]{\Tr{c}}
        {}
        ,
        \pstree[nodesep=1pt,levelsep=4ex]{\Tr{b}}
        {
            \Tr{c}
            \Tr{c}
        }
        ,
        \pstree[nodesep=1pt,levelsep=4ex]{\Tr{b}}
        {
            \Tr{a}
            \Tr{a}
        }
        ,
        \pstree[nodesep=1pt,levelsep=4ex]{\Tr{b}}
        {
            \Tr{b}
            \Tr{b}
        }
        , 
         \pstree[nodesep=1pt,levelsep=4ex]{\Tr{b}}
        {
            \Tr{a}
            \Tr{c}
        }
        ,
        \pstree[nodesep=1pt,levelsep=4ex]{\Tr{b}}
        {
            \Tr{c}
            \Tr{a}
        }       ,
        \pstree[nodesep=1pt,levelsep=4ex]{\Tr{b}}
        {
            \Tr{b}
            \Tr{c}
        }
        ,       
        \pstree[nodesep=1pt,levelsep=4ex]{\Tr{b}}
        {
            \Tr{c}
            \Tr{b}
        }
        ,
        \pstree[nodesep=1pt,levelsep=4ex]{\Tr{b}}
        {
            \Tr{a}
            \Tr{b}
        }
        ,
        \pstree[nodesep=1pt,levelsep=4ex]{\Tr{b}}
        {
            \Tr{b}
            \Tr{a}
        }

\end{Bmatrix}$}\\
Since we could generate all trees in the axioms directly from the transitions itself,  no need to set the tuple associated with the axioms.\\
Create rules for generating all trees in both $A_1$ and $A_2$.\\\\
$R$ contains rules,\\\\
%   $r_1=(\chi _1,(C_1|C_3),(C_2|C_4))$, $r_2=(\chi _2, (C_1|C_3),(C_2|C_4))$,  where \\
   $r_1=(\chi _1,U_1,U_2)$, $r_2=(\chi _2, U_1,U_2)$,  where \\
   $\chi _1=(a,-,-),\chi _2=(b,-,-)$,        
   $U_1\in \{C_1,C_3\},U_2\in \{C_2,C_4\}$\\
   $C_1 =(A_1,a,1),C_2=(A_1,a,2),C_3=(A_2,V_1,1),C_4=(A_2,V_1,2)$\\
   $V_1\in \{b,c\}$\\\\
%   $C_1 =(A_1,a,1),C_2=(A_1,a,2),C_3=(A_2,b|c,1),C_4=(A_2,b|c,2)$\\\\
So the tree insertion system equivalent to $D_{r_{3}}$ is            
$\Gamma _{r_{3}} = (\Sigma ,\mathcal{A},\mathcal{A} ',R)$ where,\\

    $\Sigma =\{a_2,b_2,c_0\}$,  
$\mathcal{A} = \{A_1,A_2\}$, $F_1=[-1,-1,-1],F_2=[-1,-1,-1]$,
$\mathcal{A}'= \{A_1\}$\\
\noindent \rule{\textwidth}{1pt}
\end{example}    
\begin{example}
%\textbf{Example 2:} $L_{r_{12}}=\{t|\mbox{ no }a\mbox{ in } t\mbox{ has an } a \mbox{as child}.\}$\\\\        
                     $L_{r_{6}}=\{t|\mbox{ no }a\mbox{ in } t\mbox{ has an } a \mbox{ as child}.\}$\\\\ 
Consider the tree automata $D_{r_{6}}=(Q,\Sigma ' ,Q_f, \Delta)$ which accepts the language $L_{r_{6}}$ over the 
alphabet $\Sigma ' =\{a_2,b_2,c_0\}$ 
where $Q=\{q_a,q_b,q_c\},Q_f=\{q_a,q_b,q_c\}$ and $\Delta$ contains following transitions.\\

\begin{tabular}{llllll}
$c\rightarrow q_c$ & $ b(q_c,q_c) \rightarrow q_b$ & $ b(q_a,q_a) \rightarrow q_b$ & $b(q_b,q_b) \rightarrow q_b$ & 
  $ b(q_a,q_c) \rightarrow q_b$\\ 
& $ b(q_c,q_a) \rightarrow q_b$ & $ b(q_b,q_c) \rightarrow q_b$ & $ b(q_c,q_b) \rightarrow q_b$ & $ b(q_a,q_b) \rightarrow q_b$\\
& $ b(q_b,q_a) \rightarrow q_b$ & $ a(q_c,q_c) \rightarrow q_a$ & $ a(q_b,q_b) \rightarrow q_a$ & $ a(q_b,q_c) \rightarrow q_a$\\
& $ a(q_c,q_b) \rightarrow q_a$\\
\end{tabular}
\\Create initial axiom $A_1$ from transitions which directly leads to final state $q_a$.

{\small $A_1=
    \begin{Bmatrix}
        \pstree[nodesep=1pt,levelsep=4ex]{\Tr{c}}
        {}
        ,
        \pstree[nodesep=1pt,levelsep=4ex]{\Tr{a}}
        {
            \Tr{c}
            \Tr{c}
        }
        ,
        \pstree[nodesep=1pt,levelsep=4ex]{\Tr{a}}
        {
            \Tr{b}
            \Tr{b}
        }
        , 
        \pstree[nodesep=1pt,levelsep=4ex]{\Tr{a}}
        {
            \Tr{b}
            \Tr{c}
        }
        ,       
        \pstree[nodesep=1pt,levelsep=4ex]{\Tr{a}}
        {
            \Tr{c}
            \Tr{b}
        }
        ,
        \pstree[nodesep=1pt,levelsep=4ex]{\Tr{b}}
        {
            \Tr{c}
            \Tr{c}
        }
        ,
        \pstree[nodesep=1pt,levelsep=4ex]{\Tr{b}}
        {
            \Tr{a}
            \Tr{a}
        }
        ,
        \pstree[nodesep=1pt,levelsep=4ex]{\Tr{b}}
        {
            \Tr{b}
            \Tr{b}
        }
        ,\\ 
        \pstree[nodesep=1pt,levelsep=4ex]{\Tr{b}}
        {
            \Tr{a}
            \Tr{c}
        }
        ,
        \pstree[nodesep=1pt,levelsep=4ex]{\Tr{b}}
        {
            \Tr{c}
            \Tr{a}
        }       ,
        \pstree[nodesep=1pt,levelsep=4ex]{\Tr{b}}
        {
            \Tr{b}
            \Tr{c}
        }
        ,       
        \pstree[nodesep=1pt,levelsep=4ex]{\Tr{b}}
        {
            \Tr{c}
            \Tr{b}
        }
        ,
        \pstree[nodesep=1pt,levelsep=4ex]{\Tr{b}}
        {
            \Tr{a}
            \Tr{b}
        }
        ,
        \pstree[nodesep=1pt,levelsep=4ex]{\Tr{b}}
        {
            \Tr{b}
            \Tr{a}
        }
     \end{Bmatrix}$}\\\\
There is no non-final state in this automata. So $\mathcal{A} =\mathcal{A} '= \{A_1\}$.
Since we could generate all trees in the axioms directly from the transitions itself, no need to set the tuple associated with the axioms.\\
Create rules for generating all trees in $A_1$.\\\\
$R$ contains rules,\\
   $r_1=(\chi _1,C_1,C_2)$,$r_2=(\chi _2, C_3,C_4)$,  where \\
   $\chi _1=(a,-,-),\chi _2=(b,-,-)$\\        
    $C_1 =(A_1,V_1,1),C_2=(A_1,V_1,2),C_3=(A_1,V_2,1),C_4=(A_1,V_2,2)$\\
        $V_1\in \{c,b\},V_2\in \{a,b,c\}$  \\\\
%    $C_1 =(A_1,c|b,1),C_2=(A_1,c|b,2),C_3=(A_1,a|b|c,1),C_4=(A_1,a|b|c,2)$\\\\      
So the tree insertion system equivalent to $D_{r_{6}}$ is            
$\Gamma _{r_{6}} = (\Sigma ,\mathcal{A},\mathcal{A} ',R)$ where,\\

    $\Sigma =\{a_2,b_2,c_0\}$,  
$\mathcal{A} = \{A_1\}$, $F_1=[-1,-1,-1]$
$\mathcal{A}'= \{A_1\}$\\
\noindent \rule{\textwidth}{1pt}
\end{example}    
\begin{example}
%\textbf{Example 4:} $L_{r_{11}}=\{t|n_a(t)\%2\neq 0\}$\\\\        
                     $L_{r_{4}}=\{t|n_a(t)\%2\neq 0\}$\\\\ 
Consider the tree automata $D_{r_{4}}=(Q,\Sigma ',Q_f, \Delta)$ which accepts the language $L_{r_{4}}$ over the alphabet 
$\Sigma '=\{a_2,b_2,c_0\}$ 
where $Q=\{e,o\},Q_f=\{o\}$ and $\Delta$ contains following transitions.\\

\begin{tabular}{llllll}
$c\rightarrow e$ & $ b(e,e) \rightarrow e$ & $ b(o,o) \rightarrow e$ & $a(e,o) \rightarrow e$ & 
$a(o,e) \rightarrow e$\\
&$b(e,o) \rightarrow o$ & $ b(o,e) \rightarrow o$ & $ a(e,e) \rightarrow o$ & $a(o,o) \rightarrow o$\\ 
\end{tabular}\\\\

Here the transition shows a recursive behaviour and so it is not possible to generate all axioms for this language.
% But inorder to design the tree insertion system, first we need some idea about the language. To get this idea, 
Generate  temporary axioms: $AF$, from final state transitions and $AN$, from non-final state transitions.

{\small 
    \begin{eqnarray*}
    \mbox{AF initially contains }
	& &
    \begin{Bmatrix}
        \pstree[nodesep=1pt,levelsep=4ex]{\Tr{b}}
        {
            \Tr{e}
            \Tr{o}
        }
        ,
        \pstree[nodesep=1pt,levelsep=4ex]{\Tr{b}}
        {
            \Tr{o}
            \Tr{e}
        }
        ,
        \pstree[nodesep=1pt,levelsep=4ex]{\Tr{a}}
        {
            \Tr{e}
            \Tr{e}
        }
        , 
         \pstree[nodesep=1pt,levelsep=4ex]{\Tr{a}}
        {
            \Tr{o}
            \Tr{o}
        }
      \end{Bmatrix} \mbox{ which leads to }\\
        %\hspace{2cm} 
%{\small 
    &  &
    \begin{Bmatrix}
        \pstree[nodesep=1pt,levelsep=4ex]{\Tr{b}}
        {
            \Tr{e}
            \Tr{o}
        }
        ,
        \pstree[nodesep=1pt,levelsep=4ex]{\Tr{b}}
        {
            \Tr{o}
            \Tr{e}
        }
        ,
        \pstree[nodesep=1pt,levelsep=4ex]{\Tr{a}}
        {
            \Tr{e}
            \Tr{e}
        }
        , 
        \pstree[nodesep=1pt,levelsep=4ex]{\Tr{a}}
        {
            \Tr{o}
            \Tr{o}
        }
        ,
	\pstree[nodesep=1pt,levelsep=4ex]{\Tr{a}}
        {}
      \end{Bmatrix} \mbox{which then leads to }\\
%{\small 
    &  &
    \begin{Bmatrix}
        \pstree[nodesep=1pt,levelsep=4ex]{\Tr{b}}
        {
            \Tr{c}
            \Tr{a}
        }
        ,
        \pstree[nodesep=1pt,levelsep=4ex]{\Tr{b}}
        {
            \Tr{b}
            \Tr{a}
        }
        ,
        \pstree[nodesep=1pt,levelsep=4ex]{\Tr{b}}
        {
            \Tr{a}
            \Tr{c}
        }
        ,
        \pstree[nodesep=1pt,levelsep=4ex]{\Tr{b}}
        {
            \Tr{a}
            \Tr{b}
        }
,
        \pstree[nodesep=1pt,levelsep=4ex]{\Tr{a}}
        {
            \Tr{c}
            \Tr{c}
        }
        , 
        \pstree[nodesep=1pt,levelsep=4ex]{\Tr{a}}
        {
            \Tr{b}
            \Tr{b}
        }
        ,
        \pstree[nodesep=1pt,levelsep=4ex]{\Tr{a}}
        {
            \Tr{c}
            \Tr{b}
        }
        ,
        \pstree[nodesep=1pt,levelsep=4ex]{\Tr{a}}
        {
            \Tr{b}
            \Tr{c}
        }
        , 
        \pstree[nodesep=1pt,levelsep=4ex]{\Tr{a}}
        {
            \Tr{a}
            \Tr{a}
        }
        , 
 \pstree[nodesep=1pt,levelsep=4ex]{\Tr{a}}
        {}
    \end{Bmatrix}\\ 
          AN \mbox{ initially contains }
&  &
      \begin{Bmatrix}
        \pstree[nodesep=1pt,levelsep=4ex]{\Tr{c}}
        {}
        ,
        \pstree[nodesep=1pt,levelsep=4ex]{\Tr{b}}
        {
            \Tr{e}
%           {\tiny{\psset{linestyle=none}\TR{(l)}} }
            \Tr{e}
        }       ,
        \pstree[nodesep=1pt,levelsep=4ex]{\Tr{b}}
        {
            \Tr{o}
            \Tr{o}
        }
        ,       
        \pstree[nodesep=1pt,levelsep=4ex]{\Tr{a}}
        {
            \Tr{e}
            \Tr{o}
        }
        ,
        \pstree[nodesep=1pt,levelsep=4ex]{\Tr{a}}
        {
            \Tr{o}
            \Tr{e}
        }
      \end{Bmatrix} \mbox{ which leads to }\\
        %\hspace{3cm}  
      &  &
      \begin{Bmatrix}
        \pstree[nodesep=1pt,levelsep=4ex]{\Tr{c}}
        {}
        ,
        \pstree[nodesep=1pt,levelsep=4ex]{\Tr{b}}
        {} 
        ,
        \pstree[nodesep=1pt,levelsep=4ex]{\Tr{b}}
        {
            \Tr{e}
            \Tr{e}
        }
	,
        \pstree[nodesep=1pt,levelsep=4ex]{\Tr{b}}
        {
            \Tr{o}
            \Tr{o}
        }
        ,       
        \pstree[nodesep=1pt,levelsep=4ex]{\Tr{a}}
        {
            \Tr{e}
            \Tr{o}
        }
        ,
        \pstree[nodesep=1pt,levelsep=4ex]{\Tr{a}}
        {
            \Tr{o}
            \Tr{e}
        }
      \end{Bmatrix} \mbox{ which then leads to}\\
%        \hspace{2cm}  
        &   &
      \begin{Bmatrix}
        \pstree[nodesep=1pt,levelsep=4ex]{\Tr{c}}
        {}
        ,
        \pstree[nodesep=1pt,levelsep=4ex]{\Tr{b}}
        {}
        ,
        \pstree[nodesep=1pt,levelsep=4ex]{\Tr{b}}
        {
            \Tr{b}
            \Tr{b}
        }
        ,  \pstree[nodesep=1pt,levelsep=4ex]{\Tr{b}}
        {
            \Tr{c}
            \Tr{c}
        }
        ,  \pstree[nodesep=1pt,levelsep=4ex]{\Tr{b}}
        {
            \Tr{b}
            \Tr{c}
        }
        ,  \pstree[nodesep=1pt,levelsep=4ex]{\Tr{b}}
        {
            \Tr{c}
            \Tr{b}
        }
        ,  \pstree[nodesep=1pt,levelsep=4ex]{\Tr{b}}
        {
            \Tr{a}
            \Tr{a}
        }
        ,       
        \pstree[nodesep=1pt,levelsep=4ex]{\Tr{a}}
        {
            \Tr{c}
            \Tr{a}
        }
        ,
        \pstree[nodesep=1pt,levelsep=4ex]{\Tr{a}}
        {
            \Tr{b}
            \Tr{a}
        }
        ,
        \pstree[nodesep=1pt,levelsep=4ex]{\Tr{a}}
        {
            \Tr{a}
            \Tr{b}
        }
        ,
        \pstree[nodesep=1pt,levelsep=4ex]{\Tr{a}}
        {
            \Tr{a}
            \Tr{c}
        }
  \end{Bmatrix}
\end{eqnarray*}
}\\

The two axiom sets $AF$ and $AN$ differs in the number of $a's$. So $AF$ has some constraint on $a$.

Set the flag associated with $AF$.

Here $x=0$.\\
Since all axioms in $AF$ contains $a$, $n=1$ and one of the axiom contains 3 $a's$, $m=3$.
So $y=1$ and $z=2$

In $AF$, split the axioms so that each axiom contains single $a's$.\\
{\small 
$    
	AF =
    \begin{Bmatrix}
        \pstree[nodesep=1pt,levelsep=4ex]{\Tr{b}}
        {
            \Tr{c}
            \Tr{a}
        }
        ,
        \pstree[nodesep=1pt,levelsep=4ex]{\Tr{b}}
        {
            \Tr{b}
            \Tr{a}
        }
        ,
        \pstree[nodesep=1pt,levelsep=4ex]{\Tr{b}}
        {
            \Tr{a}
            \Tr{c}
        }
        ,
        \pstree[nodesep=1pt,levelsep=4ex]{\Tr{b}}
        {
            \Tr{a}
            \Tr{b}
        }
,
        \pstree[nodesep=1pt,levelsep=4ex]{\Tr{a}}
        {
            \Tr{c}
            \Tr{c}
        }
        , 
        \pstree[nodesep=1pt,levelsep=4ex]{\Tr{a}}
        {
            \Tr{b}
            \Tr{b}
        }
        ,
        \pstree[nodesep=1pt,levelsep=4ex]{\Tr{a}}
        {
            \Tr{c}
            \Tr{b}
        }
        ,
        \pstree[nodesep=1pt,levelsep=4ex]{\Tr{a}}
        {
            \Tr{b}
            \Tr{c}
        }


        , 
 \pstree[nodesep=1pt,levelsep=4ex]{\Tr{a}}
        {}
    \end{Bmatrix}$}\\ 
Now let $A_1=AN$ and $A_2=AF$.\\
As both $a$ and $b$ can lead to final state directly, the insertion can be start from either $A_1$ or $A_2$.
Hence\\ 

$\mathcal{A} =\{A_1,A_2\}$. $F_1=[-1,-1,-1],F_2=[0,1,2]$
$\mathcal{A}' =\{A_1,A_2\}$.\\

Create rules for generating all trees both in $A_1$ and in $A_2$.\\
$R$ contains rules,\\
   $r_1=(\chi _1,U_1,U_2)$,
   $r_2=(\chi _2, U_1,U_2)$,
  where \\
   $\chi _1=(a,-,-),\chi _2=(b,-,-)$,
   $U_1\in \{C_1,C_3\},U_2\in \{C_2,C_4\}$\\%,U_3\in \{C_1,C_4\},U_4\in \{C_1,C_3\}$\\
    $C_1 =(A_1,V_1,1),C_2=(A_1,V_1,2),C_3=(A_2,V_2,1),C_4=(A_2,V_2,2)$\\
    $V_1\in \{c,b\},V_2\in \{a,b\}$\\\\
So the tree insertion system equivalent to $D_{r_{4}}$ is            
$\Gamma _{r_{4}} = (\Sigma ,\mathcal{A},\mathcal{A} ',R)$ where,\\

    $\Sigma =\{a_2,b_2,c_0\}$,  
$\mathcal{A} ,\mathcal{A}'$ and $R$ as given above.\\

\noindent \rule{\textwidth}{1pt}
\end{example}    





\begin{example}
%\textbf{Example 4:} $L_{r_{11}}=\{t|n_a(t)\%2\neq 0\}$\\\\        
                     $L_{r_{8}}=\{t|n_a(t)\%2= 0\}$\\\\ 
Consider the tree automata $D_{r_{8}}=(Q,\Sigma ',Q_f, \Delta)$ which accepts the language $L_{r_{8}}$ over the alphabet 
$\Sigma '=\{a_2,b_2,c_0\}$ 
where $Q=\{e,o\},Q_f=\{e\}$ and $\Delta$ contains following transitions.\\

\begin{tabular}{llllll}
$c\rightarrow e$ & $ b(e,e) \rightarrow e$ & $ b(o,o) \rightarrow e$ & $a(e,o) \rightarrow e$ & 
$a(o,e) \rightarrow e$\\
&$b(e,o) \rightarrow o$ & $ b(o,e) \rightarrow o$ & $ a(e,e) \rightarrow o$ & $a(o,o) \rightarrow o$\\ 
\end{tabular}\\\\

Here the transition shows a recursive behaviour and so it is not possible to generate all axioms for this language.
% But inorder to design the tree insertion system, first we need some idea about the language. To get this idea, 
Generate  temporary axioms: $AF$, from final state transitions and $AN$, from non-final state transitions.

{\small 
    \begin{eqnarray*}
    \mbox{AN initially contains }
	& &
    \begin{Bmatrix}
        \pstree[nodesep=1pt,levelsep=4ex]{\Tr{b}}
        {
            \Tr{e}
            \Tr{o}
        }
        ,
        \pstree[nodesep=1pt,levelsep=4ex]{\Tr{b}}
        {
            \Tr{o}
            \Tr{e}
        }
        ,
        \pstree[nodesep=1pt,levelsep=4ex]{\Tr{a}}
        {
            \Tr{e}
            \Tr{e}
        }
        , 
         \pstree[nodesep=1pt,levelsep=4ex]{\Tr{a}}
        {
            \Tr{o}
            \Tr{o}
        }
      \end{Bmatrix} \mbox{ which leads to }\\
        %\hspace{2cm} 
%{\small 
    &  &
    \begin{Bmatrix}
        \pstree[nodesep=1pt,levelsep=4ex]{\Tr{b}}
        {
            \Tr{e}
            \Tr{o}
        }
        ,
        \pstree[nodesep=1pt,levelsep=4ex]{\Tr{b}}
        {
            \Tr{o}
            \Tr{e}
        }
        ,
        \pstree[nodesep=1pt,levelsep=4ex]{\Tr{a}}
        {
            \Tr{e}
            \Tr{e}
        }
        , 
        \pstree[nodesep=1pt,levelsep=4ex]{\Tr{a}}
        {
            \Tr{o}
            \Tr{o}
        }
        ,
	\pstree[nodesep=1pt,levelsep=4ex]{\Tr{a}}
        {}
      \end{Bmatrix} \mbox{which then leads to }\\
%{\small 
    &  &
    \begin{Bmatrix}
        \pstree[nodesep=1pt,levelsep=4ex]{\Tr{b}}
        {
            \Tr{c}
            \Tr{a}
        }
        ,
        \pstree[nodesep=1pt,levelsep=4ex]{\Tr{b}}
        {
            \Tr{b}
            \Tr{a}
        }
        ,
        \pstree[nodesep=1pt,levelsep=4ex]{\Tr{b}}
        {
            \Tr{a}
            \Tr{c}
        }
        ,
        \pstree[nodesep=1pt,levelsep=4ex]{\Tr{b}}
        {
            \Tr{a}
            \Tr{b}
        }
,
        \pstree[nodesep=1pt,levelsep=4ex]{\Tr{a}}
        {
            \Tr{c}
            \Tr{c}
        }
        , 
        \pstree[nodesep=1pt,levelsep=4ex]{\Tr{a}}
        {
            \Tr{b}
            \Tr{b}
        }
        ,
        \pstree[nodesep=1pt,levelsep=4ex]{\Tr{a}}
        {
            \Tr{c}
            \Tr{b}
        }
        ,
        \pstree[nodesep=1pt,levelsep=4ex]{\Tr{a}}
        {
            \Tr{b}
            \Tr{c}
        }
        , 
        \pstree[nodesep=1pt,levelsep=4ex]{\Tr{a}}
        {
            \Tr{a}
            \Tr{a}
        }
        , 
 \pstree[nodesep=1pt,levelsep=4ex]{\Tr{a}}
        {}
    \end{Bmatrix}\\ 
          AF \mbox{ initially contains }
&  &
      \begin{Bmatrix}
        \pstree[nodesep=1pt,levelsep=4ex]{\Tr{c}}
        {}
        ,
        \pstree[nodesep=1pt,levelsep=4ex]{\Tr{b}}
        {
            \Tr{e}
%           {\tiny{\psset{linestyle=none}\TR{(l)}} }
            \Tr{e}
        }       ,
        \pstree[nodesep=1pt,levelsep=4ex]{\Tr{b}}
        {
            \Tr{o}
            \Tr{o}
        }
        ,       
        \pstree[nodesep=1pt,levelsep=4ex]{\Tr{a}}
        {
            \Tr{e}
            \Tr{o}
        }
        ,
        \pstree[nodesep=1pt,levelsep=4ex]{\Tr{a}}
        {
            \Tr{o}
            \Tr{e}
        }
      \end{Bmatrix} \mbox{ which leads to }\\
        %\hspace{3cm}  
      &  &
      \begin{Bmatrix}
        \pstree[nodesep=1pt,levelsep=4ex]{\Tr{c}}
        {}
        ,
        \pstree[nodesep=1pt,levelsep=4ex]{\Tr{b}}
        {} 
        ,
        \pstree[nodesep=1pt,levelsep=4ex]{\Tr{b}}
        {
            \Tr{e}
            \Tr{e}
        }
	,
        \pstree[nodesep=1pt,levelsep=4ex]{\Tr{b}}
        {
            \Tr{o}
            \Tr{o}
        }
        ,       
        \pstree[nodesep=1pt,levelsep=4ex]{\Tr{a}}
        {
            \Tr{e}
            \Tr{o}
        }
        ,
        \pstree[nodesep=1pt,levelsep=4ex]{\Tr{a}}
        {
            \Tr{o}
            \Tr{e}
        }
      \end{Bmatrix} \mbox{ which then leads to}\\
%        \hspace{2cm}  
        &   &
      \begin{Bmatrix}
        \pstree[nodesep=1pt,levelsep=4ex]{\Tr{c}}
        {}
        ,
        \pstree[nodesep=1pt,levelsep=4ex]{\Tr{b}}
        {}
        ,
        \pstree[nodesep=1pt,levelsep=4ex]{\Tr{b}}
        {
            \Tr{b}
            \Tr{b}
        }
        ,  \pstree[nodesep=1pt,levelsep=4ex]{\Tr{b}}
        {
            \Tr{c}
            \Tr{c}
        }
        ,  \pstree[nodesep=1pt,levelsep=4ex]{\Tr{b}}
        {
            \Tr{b}
            \Tr{c}
        }
        ,  \pstree[nodesep=1pt,levelsep=4ex]{\Tr{b}}
        {
            \Tr{c}
            \Tr{b}
        }
        ,  \pstree[nodesep=1pt,levelsep=4ex]{\Tr{b}}
        {
            \Tr{a}
            \Tr{a}
        }
        ,       
        \pstree[nodesep=1pt,levelsep=4ex]{\Tr{a}}
        {
            \Tr{c}
            \Tr{a}
        }
        ,
        \pstree[nodesep=1pt,levelsep=4ex]{\Tr{a}}
        {
            \Tr{b}
            \Tr{a}
        }
        ,
        \pstree[nodesep=1pt,levelsep=4ex]{\Tr{a}}
        {
            \Tr{a}
            \Tr{b}
        }
        ,
        \pstree[nodesep=1pt,levelsep=4ex]{\Tr{a}}
        {
            \Tr{a}
            \Tr{c}
        }
  \end{Bmatrix}
\end{eqnarray*}
}\\

The two axiom sets $AF$ and $AN$ differs in the number of $a's$. So $AF$ has some constraint on $a$.

Set the flag associated with $AF$.

Here $x=0$.\\
Since $AF$ contains some axioms without an $a$, $y=0$ and one of the axiom contains 2 $a's$, $m=2$.
So $z=1$

In $AF$, split the axioms so that each axiom contains single $a's$.\\
{\small 
$    
	AF =
    \begin{Bmatrix}
        \pstree[nodesep=1pt,levelsep=4ex]{\Tr{b}}
        {
            \Tr{c}
            \Tr{a}
        }
        ,
        \pstree[nodesep=1pt,levelsep=4ex]{\Tr{b}}
        {
            \Tr{b}
            \Tr{a}
        }
        ,
        \pstree[nodesep=1pt,levelsep=4ex]{\Tr{b}}
        {
            \Tr{a}
            \Tr{c}
        }
        ,
        \pstree[nodesep=1pt,levelsep=4ex]{\Tr{b}}
        {
            \Tr{a}
            \Tr{b}
        }
,
        \pstree[nodesep=1pt,levelsep=4ex]{\Tr{a}}
        {
            \Tr{c}
            \Tr{c}
        }
        , 
        \pstree[nodesep=1pt,levelsep=4ex]{\Tr{a}}
        {
            \Tr{b}
            \Tr{b}
        }
        ,
        \pstree[nodesep=1pt,levelsep=4ex]{\Tr{a}}
        {
            \Tr{c}
            \Tr{b}
        }
        ,
        \pstree[nodesep=1pt,levelsep=4ex]{\Tr{a}}
        {
            \Tr{b}
            \Tr{c}
        }


        , 
 \pstree[nodesep=1pt,levelsep=4ex]{\Tr{a}}
        {}
    \end{Bmatrix}$}\\ 
Now let $A_1=AN$ and $A_2=AF$.\\
As both $a$ and $b$ can lead to final state directly, the insertion can be start from either $A_1$ or $A_2$.
Hence\\ 

$\mathcal{A} =\{A_1,A_2\}$. $F_1=[-1,-1,-1],F_2=[0,0,1]$
$\mathcal{A}' =\{A_1,A_2\}$.\\

Create rules for generating all trees both in $A_1$ and in $A_2$.\\
$R$ contains rules,\\
   $r_1=(\chi _1,U_1,U_2)$,
   $r_2=(\chi _2, U_1,U_2)$,
  where \\
   $\chi _1=(a,-,-),\chi _2=(b,-,-)$,
   $U_1\in \{C_1,C_3\},U_2\in \{C_2,C_4\}$\\%,U_3\in \{C_1,C_4\},U_4\in \{C_1,C_3\}$\\
    $C_1 =(A_1,V_1,1),C_2=(A_1,V_1,2),C_3=(A_2,V_2,1),C_4=(A_2,V_2,2)$\\
    $V_1\in \{c,b\},V_2\in \{a,b\}$\\\\
So the tree insertion system equivalent to $D_{r_{8}}$ is            
$\Gamma _{r_{8}} = (\Sigma ,\mathcal{A},\mathcal{A} ',R)$ where,\\

    $\Sigma =\{a_2,b_2,c_0\}$,  
$\mathcal{A} ,\mathcal{A}'$ and $R$ as given above.\\

\noindent \rule{\textwidth}{1pt}
\end{example}    
\emph{\textbf{Result 2:} Given a finite bottom-up tree automata, we can construct an equivalent tree insertion system.}\\
\noindent \rule{\textwidth}{1pt}\\\\
\emph{\textbf{Claim: From Result 1 and Result 2, the generative powers of tree insertion systems and regular tree grammars are equal.}}\\
\subsection{Extended Systems}
An \emph{extended tree insertion system} can be defined by defining two types of alphabets $T$ and $N$ with $\Sigma = T\cup N$ 
and additional restriction that
the symbols in $T$ have arity 0 and each symbol from $N$ has a finite number of arities.

With this extended definition, the parse trees of a context-free grammar can be generated. 
\begin{example}
Consider the $CFG$ 
\hbox{$G=(\{S\},\{a,b\},\{S\rightarrow aSS,S\rightarrow b\},S)$}. The parse tree of $G$ can be generated by the extended tree insertion system
$(\Sigma ,\mathcal{A} ,\mathcal{A} ',R)$ with $\Sigma =N\cup T$, $N=\{S\}$, $T=\{a,b\}$, arity of $S$ is $\{1,3\}$. Axioms are 
{\small
$A_1=\{S
%\begin{Bmatrix}
 %       \pstree[nodesep=1pt,levelsep=4ex]{\Tr{S}}
  %      { }
%\end{Bmatrix}
\}
$}
and
{\small
$
A_2=\{
%\begin{Bmatrix}
  %      \pstree[nodesep=1pt,levelsep=4ex]{\Tr{a}}
   %     { }
	a
%	,
%	\pstree[nodesep=1pt,levelsep=4ex]{\Tr{S}}
 %       { }
%	S
	,
%	\pstree[nodesep=1pt,levelsep=4ex]{\Tr{b}}
 %       { }
	b
%\end{Bmatrix}
\}
$}
with flags $F_1=[-1,-1,-1],F_2=[-1,-1,-1]$, where $\mathcal{A} '=\{A_1\}$. The rules are $r_1=(\chi _1 ,C_1,C_2,C_3)$ and $r_2=(\chi _1,C_4)$, 
where $\chi _1=(S,-,-),C_1=(A_2,a,1),C_2=(A_1,S,2),C_3=(A_1,S,3),C_4=(A_2,b,1)$.

As an example of derivation

\hbox{
{\tiny
$
	\pstree[nodesep=1pt,levelsep=4ex]{\Tr{S}}
        { 
	}
 \Rightarrow       
	\pstree[nodesep=1pt,levelsep=4ex]{\Tr{S}}
        { 
		\Tr{a}
		\Tr{S}		
		\Tr{S}
	}
\Rightarrow
	\pstree[nodesep=1pt,levelsep=4ex]{\Tr{S}}
        { 
		\Tr{a}
		\pstree[nodesep=1pt,levelsep=4ex]{\Tr{S}}
	        {
			\Tr{b}
		} 
		\Tr{S}
	}
\Rightarrow
	\pstree[nodesep=1pt,levelsep=4ex]{\Tr{S}}
        { 
		\Tr{a}
		\pstree[nodesep=1pt,levelsep=4ex]{\Tr{S}}
	        {
			\Tr{b}
		} 
		\pstree[nodesep=1pt,levelsep=4ex]{\Tr{S}}
		{ 
			\Tr{a}
			\Tr{S}
			\Tr{S}
		}
	}
\Rightarrow
	\pstree[nodesep=1pt,levelsep=4ex]{\Tr{S}}
        { 
		\Tr{a}
		\pstree[nodesep=1pt,levelsep=4ex]{\Tr{S}}
	        {
			\Tr{b}
		} 
		\pstree[nodesep=1pt,levelsep=4ex]{\Tr{S}}
		{ 
			\Tr{a}
			\pstree[nodesep=1pt,levelsep=4ex]{\Tr{S}}
		        {
				\Tr{b}
			} 
			\Tr{S}
		}
	}
\Rightarrow
	\pstree[nodesep=1pt,levelsep=4ex]{\Tr{S}}
        { 
		\Tr{a}
		\pstree[nodesep=1pt,levelsep=4ex]{\Tr{S}}
	        {
			\Tr{b}
		} 
		\pstree[nodesep=1pt,levelsep=4ex]{\Tr{S}}
		{ 
			\Tr{a}
			\pstree[nodesep=1pt,levelsep=4ex]{\Tr{S}}
		        {
				\Tr{b}
			} 
			\pstree[nodesep=1pt,levelsep=4ex]{\Tr{S}}
	        	{
				\Tr{b}
			} 
		}
	}
$}}
\noindent \rule{\textwidth}{1pt}
\end{example}





\begin{example}
Consider the $CFG$ 
\hbox{$G=(\{S\},\{a,b\},\{S\rightarrow aSb,S\rightarrow SS\},S)$}. The parse tree of $G$ can be generated by the extended tree insertion system
$(\Sigma ,\mathcal{A} ,\mathcal{A} ',R)$ with $\Sigma =N\cup T$, $N=\{S\}$, $T=\{a,b\}$, arity of $S$ is $\{2,3\}$. Axioms are 
{\small
$A_1=\{S
%\begin{Bmatrix}
 %       \pstree[nodesep=1pt,levelsep=4ex]{\Tr{S}}
  %      { }
%\end{Bmatrix}
\}
$}
and
{\small
$
A_2=\{
%\begin{Bmatrix}
  %      \pstree[nodesep=1pt,levelsep=4ex]{\Tr{a}}
   %     { }
	a
%	,
%	\pstree[nodesep=1pt,levelsep=4ex]{\Tr{S}}
 %       { }
%	S
	,
%	\pstree[nodesep=1pt,levelsep=4ex]{\Tr{b}}
 %       { }
	b
%\end{Bmatrix}
\}
$}
with flags $F_1=[-1,-1,-1],F_2=[-1,-1,-1]$, where $\mathcal{A} '=\{A_1\}$. The rules are $r_1=(\chi _1 ,C_1,C_2,C_3)$ , $r_2=(\chi _1,C_1,C_2)$ 
and $r_3=(\chi _1,C_4,C_5)$, 
where $\chi _1=(S,-,-),C_1=(A_2,a,1),C_2=(A_1,S,2),C_3=(A_2,b,3),C_4=(A_1,S,1),C_5=(A_1,S,2)$.

As an example of derivation

\hbox{
{\small
$
	\pstree[nodesep=1pt,levelsep=4ex]{\Tr{S}}
        { 
	}
 \Rightarrow       
	\pstree[nodesep=1pt,levelsep=4ex]{\Tr{S}}
        { 
		\Tr{a}
		\Tr{S}		
		\Tr{b}
	}
\Rightarrow
	\pstree[nodesep=1pt,levelsep=4ex]{\Tr{S}}
        { 
		\Tr{a}
		\pstree[nodesep=1pt,levelsep=4ex]{\Tr{S}}
	        {
			\Tr{S}
			\Tr{S}
		} 
		\Tr{b}
	}
\Rightarrow
	\pstree[nodesep=1pt,levelsep=4ex]{\Tr{S}}
        { 
		\Tr{a}
		\pstree[nodesep=1pt,levelsep=4ex]{\Tr{S}}
	        {
			\pstree[nodesep=1pt,levelsep=4ex]{\Tr{S}}
			{
				\Tr{a}
				\Tr{b}
			}
			\pstree[nodesep=1pt,levelsep=4ex]{\Tr{S}}
			{
				\Tr{S}
				\Tr{S}
			}	
		} 
		\Tr{b}
	}
\Rightarrow
	\pstree[nodesep=1pt,levelsep=4ex]{\Tr{S}}
        { 
		\Tr{a}
		\pstree[nodesep=1pt,levelsep=4ex]{\Tr{S}}
	        {
			\pstree[nodesep=1pt,levelsep=4ex]{\Tr{S}}
			{
				\Tr{a}
				\Tr{b}
			}
			\pstree[nodesep=1pt,levelsep=4ex]{\Tr{S}}
			{
				\pstree[nodesep=1pt,levelsep=4ex]{\Tr{S}}
				{
					\Tr{a}
					\Tr{b}
				}
				\pstree[nodesep=1pt,levelsep=4ex]{\Tr{S}}
				{
					\Tr{a}
					\Tr{b}
				}
			}	
		} 
		\Tr{b}
	}
$}}
\noindent \rule{\textwidth}{1pt}
\end{example}

But the extended tree insertion system is not capable of generating parse trees for some context free grammars like 
$G=({S,B,C},{a,b},{S\rightarrow aB|bC|aa|bb,B\rightarrow Sa,C\rightarrow Sb},S)$. 

The extended tree insertion system $\Gamma=(\Sigma,\mathcal{A},\mathcal{A'},R)$, where

 $\Sigma=\{S,B,C,a,b\}$, $\mathcal{A}=\{A_1,A_2\}$, $\mathcal{A'}=\{A_1\}$\\
{\small
$A_1=
\begin{Bmatrix}
        \pstree[nodesep=1pt,levelsep=4ex]{\Tr{S}}
         { 
		\Tr{a}
		\Tr{B}	
	 }
	,
	\pstree[nodesep=1pt,levelsep=4ex]{\Tr{S}}
         { 
		\Tr{b}
		\Tr{C}	
	 }
	,
	\pstree[nodesep=1pt,levelsep=4ex]{\Tr{S}}
         { 
		\Tr{a}
		\Tr{a}	
	 }
	,
	\pstree[nodesep=1pt,levelsep=4ex]{\Tr{S}}
         { 
		\Tr{b}
		\Tr{b}	
	 }
\end{Bmatrix}
$}
,
{\small
$
A_2=
\begin{Bmatrix}
      \pstree[nodesep=1pt,levelsep=4ex]{\Tr{B}}
         { 
		\Tr{S}
		\Tr{a}	
	 }
	,
	\pstree[nodesep=1pt,levelsep=4ex]{\Tr{C}}
         { 
		\Tr{S}
		\Tr{b}	
	 }
	,
	\pstree[nodesep=1pt,levelsep=4ex]{\Tr{a}}
         { 		
	 }
	,
	\pstree[nodesep=1pt,levelsep=4ex]{\Tr{b}}
         { 
	 }
\end{Bmatrix}
$}

 and $R=\{r_1,r_2,r_3\}$ where,\\$r_1=(\chi_1,C_1,C_2),r_2=(\chi_2,C_3,C_4),r_3=(\chi_3,C_3,C_5)$\\
$\chi_1=(S,-,-),\chi_2=(B,-,-),\chi_3=(C,-,-)\\C_1=(A_2,U_1,1),C_2=(A_2,U_2,2),C_3=(A_1,S,1),C_4=(A_2,a,2),C_5=(A_2,b,2)\\
U_1\in\{(a,b\},U_2\in\{B,C,a,b\}$\\
can generate tree in figure~\ref{xtnd_prstree1} which is not generated by the given grammar $G$.
\begin{figure}[h]
\begin{center}
{\small
      \pstree[nodesep=1pt,levelsep=4ex]{\Tr{S}}
         { 
		\Tr{a}
		\pstree[nodesep=1pt,levelsep=4ex]{\Tr{B}}
         	{ 
			\pstree[nodesep=1pt,levelsep=4ex]{\Tr{S}}	
			{
				\Tr{b}
				\pstree[nodesep=1pt,levelsep=4ex]{\Tr{B}}
				{
					\pstree[nodesep=1pt,levelsep=4ex]{\Tr{S}}	
					{	
						\Tr{a}	
						\Tr{a}	
					}
					\Tr{a}	
				}
			}
			\Tr{a}	
		}
	 }
}
\end{center}
\label{xtnd_prstree1}
\caption{Parse Tree Generated}
\end{figure}
By using a tree insertion system with replacement, we can generate the parse tree of the above grammar.
\subsection{Tree Insertion Systems with Replacement}
A \emph{tree insertion system with replacement} can be defined by defining two types of alphabets $T$ and $N$ with $\Sigma = T\cup N$ 
and additional restriction that
the symbols in $T$ have arity 0 and each symbol from $N$ has a finite number of arities. Here a subtree is inserted in to a tree by cutting the
 tree at an internal node and inserting the subtree with the cut subtree attaching to a leaf of the inserted tree. 
\begin{example}
The  tree insertion system with replacement $\Gamma=(\Sigma,\mathcal{A},\mathcal{A'},R)$, where

 $\Sigma=\{S,B,C,a,b\}$, $\mathcal{A}=\{A_1,A_2\}$, $\mathcal{A'}=\{A_1\}$\\
{\small
$A_1=
\begin{Bmatrix}
        \pstree[nodesep=1pt,levelsep=4ex]{\Tr{S}}
         { 
		\Tr{a}
		\Tr{B}	
	 }
	,
	\pstree[nodesep=1pt,levelsep=4ex]{\Tr{S}}
         { 
		\Tr{b}
		\Tr{C}	
	 }
	,
	\pstree[nodesep=1pt,levelsep=4ex]{\Tr{S}}
         { 
		\Tr{a}
		\Tr{a}	
	 }
	,
	\pstree[nodesep=1pt,levelsep=4ex]{\Tr{S}}
         { 
		\Tr{b}
		\Tr{b}	
	 }
\end{Bmatrix}
$}
,
{\small
$
A_2=
\begin{Bmatrix}
      \pstree[nodesep=1pt,levelsep=4ex]{\Tr{B}}
         { 
		\Tr{S}
		\Tr{a}	
	 }
	,
	\pstree[nodesep=1pt,levelsep=4ex]{\Tr{C}}
         { 
		\Tr{S}
		\Tr{b}	
	 }
	,
	\pstree[nodesep=1pt,levelsep=4ex]{\Tr{a}}
         { 		
	 }
	,
	\pstree[nodesep=1pt,levelsep=4ex]{\Tr{b}}
         { 
	 }
\end{Bmatrix}
$}

 and $R=\{r_1,r_2,r_3\}$ where,\\$r_1=(\chi_1,C_1),r_2=(\chi_2,C_2),r_3=(\chi_3,C_3),r_4=(\chi_4,C_3)$\\
$\chi_1=(S,-,C),\chi_2=(S,-,B),\chi_3=(B,S,-),\chi_4=(C,S,-)\\C_1=(A_2,C,2),C_2=(A_2,B,2),C_3=(A_1,S,1)\\$
As an example of derivation
\begin{center}
\hbox{
{\small
$
	\pstree[nodesep=1pt,levelsep=4ex]{\Tr{S}}
        {
		\Tr{a} 
		\Tr{B}
	}
\Rightarrow
	\pstree[nodesep=1pt,levelsep=4ex]{\Tr{S}}
        { 
		\Tr{a}
		\pstree[nodesep=1pt,levelsep=4ex]{\Tr{B}}
	        {\psset{linestyle=dashed}
			\Tr{S}
			\Tr{a}
		} 
	}
\Rightarrow
	\pstree[nodesep=1pt,levelsep=4ex]{\Tr{S}}
        { 
		\Tr{a}
		\pstree[nodesep=1pt,levelsep=4ex]{\Tr{B}}
	        {
			\pstree[nodesep=1pt,levelsep=4ex]{\Tr{S}}
			{\psset{linestyle=dashed}
				\Tr{b}
				\Tr{C}
			}
			\Tr{a}
		} 
	}
\Rightarrow
	\pstree[nodesep=1pt,levelsep=4ex]{\Tr{S}}
        { 
		\Tr{a}
		\pstree[nodesep=1pt,levelsep=4ex]{\Tr{B}}
	        {
			\pstree[nodesep=1pt,levelsep=4ex]{\Tr{S}}
			{
				\Tr{b}
				\pstree[nodesep=1pt,levelsep=4ex]{\Tr{C}}
				{\psset{linestyle=dashed}
					\Tr{S}
					\Tr{b}
				}
			}
			\Tr{a}
		} 
	}
\Rightarrow
	\pstree[nodesep=1pt,levelsep=4ex]{\Tr{S}}
        { 
		\Tr{a}
		\pstree[nodesep=1pt,levelsep=4ex]{\Tr{B}}
	        {
			\pstree[nodesep=1pt,levelsep=4ex]{\Tr{S}}
			{
				\Tr{b}
				\pstree[nodesep=1pt,levelsep=4ex]{\Tr{C}}
				{
					\pstree[nodesep=1pt,levelsep=4ex]{\Tr{S}}
					{\psset{linestyle=dashed}
						\Tr{a}
						\Tr{a}
					}
					\Tr{b}
				}
			}
			\Tr{a}
		} 
	}
\Rightarrow \cdots \Rightarrow 
	\pstree[nodesep=1pt,levelsep=4ex]{\Tr{S}}
        { 
		\Tr{a}
		\pstree[nodesep=1pt,levelsep=4ex]{\Tr{B}}
	        {
			\pstree[nodesep=1pt,levelsep=4ex]{\Tr{S}}
			{
				\psset{linestyle=dashed}
				\Tr{a}
				\pstree[nodesep=1pt,levelsep=4ex]{\Tr{B}}
	        		{
					\pstree[nodesep=1pt,levelsep=4ex]{\Tr{S}}
					{\psset{linestyle=solid}	
						\Tr{b}
						\pstree[nodesep=1pt,levelsep=4ex]{\Tr{C}}
						{
							\pstree[nodesep=1pt,levelsep=4ex]{\Tr{S}}
							{
								\Tr{a}
								\Tr{a}
							}
							\Tr{b}
						}
					}
					\Tr{a}
				} 
			}
			\Tr{a}
		}
	}
$}}
\noindent \rule{\textwidth}{1pt}
\end{center}
\end{example}
%

\section{Blocked Tree Insertion System}
We know that \emph{tree insertion system} can be used to generate any regular tree languages. Since it has control only at the part
of tree where current insertion taking place, it cannot be used for generating non-regular tree languages. But we can increase the 
generating power of the system by grouping the rules together to form a block and the resulting system is known as \emph{blocked 
tree insertion system}.Here a block contains one or more rules of insertion. If a block starts working, all the rules in that block must 
execute before giving control to some other block. 

\begin{definition}

The \emph{tree insertion system} is a tuple
\\ $\Gamma_B = (\Sigma,\mathcal{A},\mathcal{A} ',R)$ where,
\begin{itemize}
    \item $\Sigma$ is a finite set of \emph{ranked alphabets}.
    \item $\mathcal{A} = \{A_1\cup A_2\cup \cdots \cup A_m\}$, where each $A_i, 1\leq i\leq m$ is a finite set of axioms.\\
	With each set $A_i$ is associated  a \emph{\textbf{flag}} $F_i$ which is a triple $[x_i,y_i,z_i]$, 
	where $x_i,y_i,z_i\in\{-1,0,1,\cdots ,k\}$, for some fixed $k$, are integers,  
	which plays some role in language generation 
   	 unless $x_i=y_i=z_i=-1$. (For each insertion from $A_i$, the $x_i$ value gets incremented if $x_i\leq y_i$ and 
   	 the $x_i$ value gets decremented if $x_i>y_i$. The $x_i$ will be set to $z_i$ if one insertion from $A_i$ 
   	 happens when $x_i=y_i$. The tree insertion system is said to be \emph{\textbf{stable}}, if $x_i=y_i$ for all flags with 
   	 $x_i\leq y_i$ initially and $x_i\neq y_i$ for all flags with $x_i>y_i$ initially.)
    \item $\mathcal{A} '\subseteq \mathcal{A}$ is a finite set of \emph{initial axioms}.
    \item $R=\{B_1,B_2,\cdots,B_m\}$ is a finite set of block of insertion rules\\\\
		$B_i's$ represents block of rules in the form $\{r_1,r_2,\cdots ,r_n\}$ \\
   		 Each $r_i$, for $1\leq i\leq n$ is of the form $(\chi, C_1,C_2,\cdots ,C_p)$ where, 

   	 \begin{itemize}    
        	 \item $\chi = (root,left,right)$ which represents a context.
	
       			 \begin{itemize}   
 
          			  \item $root$  is any node in the tree
            			\item $left $ is $i^{th}$ child of $root$.% (- checks for the absence of $left$).% $1\leq i< arity(rt)$
            			\item $right$ is  $(i+1)^{th}$ child of $root$.   $0\le i \le arity(root)$ and $p\leq arity(root)$\\
				($-$ checks for the absence of a child). %$0\leq i< arity(rt)$
                 	 \end{itemize}
       		 \item $C_i = (X,rt',k),1\leq i\leq p,X\in \mathcal{A}$ 
	             	\begin{itemize}    
%              			  \item $A_j$  : $j^{th}$ axiom. $1\leq j\leq m$
               			 \item $rt'$ is the root of the tree to be attached.%$rt'\notin F$
               			 \item $k$ is the position at which $rt'$ is to get attached. $1\le k \leq arity(root)$ and it is between the nodes $left$ 							and $right$.
           		  \end{itemize}
          \end{itemize}

	\end{itemize}
\end{definition}


The \textbf{language generated} by a  blocked tree insertion system $\Gamma_B$, represented by $L(\Gamma_B)$, is the set of trees,  
with each node having children exactly equal to its arity, derivable in $\Gamma_B$, when it is in \emph{stable stae},
 from an initial axiom, using rules of $\Gamma_B$.
%all nodes has children exactly as its arity, derivable by $\Gamma_B$ from any of its initial axiom.
\\\\
$L(\Gamma_B)=\begin{Bmatrix} t|S\overset{*}\Rightarrow t, S \mbox{ in some } A_i\in \mathcal{A} '. \mbox{ Each node of }t 
\mbox{ has children exactly as its arity}\\ \mbox{ and } \Gamma_B \mbox{ is in stable state}\end{Bmatrix}$\\

\begin{example}$L_{nr_1}=\{a(b^i(g),c^i(g)),i\ge 1\}$\\\\
$\Gamma_{B _{nr_1}}= (\Sigma,\mathcal{A},\mathcal{A} ',R)$ where,\\
$\Sigma=\{a_2,b_1,c_1,g_0\}$, $\mathcal{A}=\{A_1,A_2\}$ ,$\mathcal{A}' =\{A_1\}$\\
{\small $A_1=
\begin{Bmatrix}
        \pstree[nodesep=1pt,levelsep=3ex]{\Tr{a}}
        {
            \Tr{b}
            \Tr{c}
        }
        \end{Bmatrix}
        ,
        \hspace{2cm}    
        A_2=
     \begin{Bmatrix}       
        \pstree[nodesep=1pt,levelsep=6ex]{\Tr{b}}
        {}
        ,
        \pstree[nodesep=1pt,levelsep=6ex]{\Tr{c}}
        {}
        ,
        \pstree[nodesep=1pt,levelsep=6ex]{\Tr{g}}
        {}
\end{Bmatrix}$}
\\\\$R=\{B_1,B_2\}$, where\\\\
   $B_1=\{r_1,r_2\}$ and $B_2=\{r_3,r_4\}$\\
   $r_1=(\chi _1,C_1)$,   $r_2=(\chi _2,C_2)$,  $r_3=(\chi _1,C_3)$, $r_4=(\chi _2,C_3)$, where \\
   $\chi _1=(b,-,-)$,   $\chi _2=(c,-,-)$\\   $C_1=(A_2,b,1)$,     $C_2=(A_2,c,1)$,   $C_3=(A_2,g,1)$\\
\noindent \rule{\textwidth}{1pt}
\end{example}
\begin{example} $L_{nr_2}=\{a(b^i(g),c^i(g),d^i(g))\},i\ge 1$\\\\
$\Gamma_{B _{nr_2}}= (\Sigma,\mathcal{A},\mathcal{A} ',R)$ where,\\
$\Sigma=\{a_2,b_1,c_1,d_1,g_0\}$,   $\mathcal{A} =\{A_1,A_2\}$ ,$\mathcal{A} ' =\{A_1\}$\\
{\small $A_1=
\begin{Bmatrix}
        \pstree[nodesep=1pt,levelsep=3ex]{\Tr{a}}
        {
            \Tr{b}
            \Tr{c}
            \Tr{d}
        }
        \end{Bmatrix}
        ,
        \hspace{2cm}    
        A_2=
     \begin{Bmatrix}       
        \pstree[nodesep=1pt,levelsep=6ex]{\Tr{b}}
        {}
        ,
        \pstree[nodesep=1pt,levelsep=6ex]{\Tr{c}}
        {}
        ,
        \pstree[nodesep=1pt,levelsep=6ex]{\Tr{d}}
        {}
        ,
        \pstree[nodesep=1pt,levelsep=6ex]{\Tr{g}}
        {}
\end{Bmatrix}$}
\\\\$R=\{B_1,B_2\}$, where\\\\
   $B_1=\{r_1,r_2,r_3\}$ and $B_2=\{r_4,r_5,r_6\}$\\
   $r_1=(\chi _1,C_1)$,   $r_2=(\chi _2,C_2)$,  $r_3=(\chi _3,C_3)$, $r_4=(\chi _1,C_4)$
   $r_5=(\chi _1,C_4)$,  $r_6=(\chi _1,C_4)$ ,where\\ 
   $\chi _1=(b,-,-)$,   $\chi _2=(c,-,-)$,  $\chi _3=(d,-,-)$\\   $C_1=(A_2,b,1)$,  $C_2=(A_2,c,1)$,  $C_3=(A_2,d,1)$,  $C_4=(A_2,g,1)$\\
\noindent \rule{\textwidth}{1pt}
\end{example}



{%\small
\begin{thebibliography}{99}\setlength{\itemsep}{-2mm}
	\bibitem{akihiro} Akihiro Takaharai,. Takashi Yokomori, On the Computational power of insertion-deletion systems, Natural
			Computing 2. pp. 321-336. Kluwer Academic Publishers,(2003)
	\bibitem{paun} Gheorghe Paun., Grzegorz Rozenberg., Arto Saloma,. DNA Computing, New Computing Paradigms. Springer, (1998)	
	\bibitem{lila} Lila Kari., Georghe Paun., Thierrin,G., Yu,S., At the crooroads of DNA computing and formal languages: Characterizing
			RE using insertion-deletion systems. Proc. of $3^{rd}$ DIMACS Workshop on DNA based Computing, Philadelphia, pp. 318-333, (1997)
	\bibitem{hubert} Hubert Comon., Max Dauchet Remi Gilleron.,Florent Jacquemard Denis Lugiez., Christof Loding., Sophie Tison Marc Tommasi., 
			Tree Automata Techniques and Applications. Draft book; available electronically on http://www.grappa.univ-lille3.fr/tata, 2008.
	\bibitem{kamala} M B Siddardha and K. Krithivasan, Ambiguity in Insertion-Deletion Kolam Array Grammers ,
			Journal of Combinatorics, Information and System Sciences, Vol. 33,Nos. 3-4, pages 323-337, 2008.
	\bibitem{murata}M. Murata, D. Lee, M. Mani and K. Kawaguchi. Taxonomy of XML Schema Languages using Formal Language Theory, \emph{ACM Trans.
	\bibitem{marcus}S. Marcus, Contextual grammars, Rev. Roum. Math. Pures Appl., 14 (1969),1525–1534.
	\bibitem{margentern} M. Margenstern, Gh. Paun, Yu. Rogozhin, S. Verlan, Context-free insertion-deletion systems. Theoretical Computer Science, 
			330 (2005), 339–348. Inter. Tech.}, 5(4):660-704, 2005.
	\bibitem{verlan}Sergey Verlan, On Minimal Context-Free Insertion-Deletion Systems,Journal of Automata, Languages and Combinatorics,2007,
			12,317-328,1-2.
	\bibitem{madhu}Madhu Mutyam and Kamala Krithivasan and A. Siddhartha Reddy,On Characterizing Recursively Enumerable Languages by Insertion
               	Grammars,Fundam. Inform.,64,1-4,2005,317-324,

\end{thebibliography}
}
\newpage
%\pagenumbering{num_style}{Roman}
%\setcounter{page}{1}



