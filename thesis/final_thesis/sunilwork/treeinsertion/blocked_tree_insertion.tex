

\section{Blocked Tree Insertion System}
We know that \emph{tree insertion system} can be used to generate any regular tree languages. Since it has control only at the part
of tree where current insertion taking place, it cannot be used for generating non-regular tree languages. But we can increase the 
generating power of the system by grouping the rules together to form a block and the resulting system is known as \emph{blocked 
tree insertion system}.Here a block contains one or more rules of insertion. If a block starts working, all the rules in that block must 
execute before giving control to some other block. 

\begin{definition}

The \emph{tree insertion system} is a tuple
\\ $\Gamma_B = (\Sigma,\mathcal{A},\mathcal{A} ',R)$ where,
\begin{itemize}
    \item $\Sigma$ is a finite set of \emph{ranked alphabets}.
    \item $\mathcal{A} = \{A_1\cup A_2\cup \cdots \cup A_m\}$, where each $A_i, 1\leq i\leq m$ is a finite set of axioms.\\
	With each set $A_i$ is associated  a \emph{\textbf{flag}} $F_i$ which is a triple $[x_i,y_i,z_i]$, 
	where $x_i,y_i,z_i\in\{-1,0,1,\cdots ,k\}$, for some fixed $k$, are integers,  
	which plays some role in language generation 
   	 unless $x_i=y_i=z_i=-1$. (For each insertion from $A_i$, the $x_i$ value gets incremented if $x_i\leq y_i$ and 
   	 the $x_i$ value gets decremented if $x_i>y_i$. The $x_i$ will be set to $z_i$ if one insertion from $A_i$ 
   	 happens when $x_i=y_i$. The tree insertion system is said to be \emph{\textbf{stable}}, if $x_i=y_i$ for all flags with 
   	 $x_i\leq y_i$ initially and $x_i\neq y_i$ for all flags with $x_i>y_i$ initially.)
    \item $\mathcal{A} '\subseteq \mathcal{A}$ is a finite set of \emph{initial axioms}.
    \item $R=\{B_1,B_2,\cdots,B_m\}$ is a finite set of block of insertion rules\\\\
		$B_i's$ represents block of rules in the form $\{r_1,r_2,\cdots ,r_n\}$ \\
   		 Each $r_i$, for $1\leq i\leq n$ is of the form $(\chi, C_1,C_2,\cdots ,C_p)$ where, 

   	 \begin{itemize}    
        	 \item $\chi = (root,left,right)$ which represents a context.
	
       			 \begin{itemize}   
 
          			  \item $root$  is any node in the tree
            			\item $left $ is $i^{th}$ child of $root$.% (- checks for the absence of $left$).% $1\leq i< arity(rt)$
            			\item $right$ is  $(i+1)^{th}$ child of $root$.   $0\le i \le arity(root)$ and $p\leq arity(root)$\\
				($-$ checks for the absence of a child). %$0\leq i< arity(rt)$
                 	 \end{itemize}
       		 \item $C_i = (X,rt',k),1\leq i\leq p,X\in \mathcal{A}$ 
	             	\begin{itemize}    
%              			  \item $A_j$  : $j^{th}$ axiom. $1\leq j\leq m$
               			 \item $rt'$ is the root of the tree to be attached.%$rt'\notin F$
               			 \item $k$ is the position at which $rt'$ is to get attached. $1\le k \leq arity(root)$ and it is between the nodes $left$ 							and $right$.
           		  \end{itemize}
          \end{itemize}

	\end{itemize}
\end{definition}


The \textbf{language generated} by a  blocked tree insertion system $\Gamma_B$, represented by $L(\Gamma_B)$, is the set of trees,  
with each node having children exactly equal to its arity, derivable in $\Gamma_B$, when it is in \emph{stable stae},
 from an initial axiom, using rules of $\Gamma_B$.
%all nodes has children exactly as its arity, derivable by $\Gamma_B$ from any of its initial axiom.
\\\\
$L(\Gamma_B)=\begin{Bmatrix} t|S\overset{*}\Rightarrow t, S \mbox{ in some } A_i\in \mathcal{A} '. \mbox{ Each node of }t 
\mbox{ has children exactly as its arity}\\ \mbox{ and } \Gamma_B \mbox{ is in stable state}\end{Bmatrix}$\\

\begin{example}$L_{nr_1}=\{a(b^i(g),c^i(g)),i\ge 1\}$\\\\
$\Gamma_{B _{nr_1}}= (\Sigma,\mathcal{A},\mathcal{A} ',R)$ where,\\
$\Sigma=\{a_2,b_1,c_1,g_0\}$, $\mathcal{A}=\{A_1,A_2\}$ ,$\mathcal{A}' =\{A_1\}$\\
{\small $A_1=
\begin{Bmatrix}
        \pstree[nodesep=1pt,levelsep=3ex]{\Tr{a}}
        {
            \Tr{b}
            \Tr{c}
        }
        \end{Bmatrix}
        ,
        \hspace{2cm}    
        A_2=
     \begin{Bmatrix}       
        \pstree[nodesep=1pt,levelsep=6ex]{\Tr{b}}
        {}
        ,
        \pstree[nodesep=1pt,levelsep=6ex]{\Tr{c}}
        {}
        ,
        \pstree[nodesep=1pt,levelsep=6ex]{\Tr{g}}
        {}
\end{Bmatrix}$}
\\\\$R=\{B_1,B_2\}$, where\\\\
   $B_1=\{r_1,r_2\}$ and $B_2=\{r_3,r_4\}$\\
   $r_1=(\chi _1,C_1)$,   $r_2=(\chi _2,C_2)$,  $r_3=(\chi _1,C_3)$, $r_4=(\chi _2,C_3)$, where \\
   $\chi _1=(b,-,-)$,   $\chi _2=(c,-,-)$\\   $C_1=(A_2,b,1)$,     $C_2=(A_2,c,1)$,   $C_3=(A_2,g,1)$\\
\noindent \rule{\textwidth}{1pt}
\end{example}
\begin{example} $L_{nr_2}=\{a(b^i(g),c^i(g),d^i(g))\},i\ge 1$\\\\
$\Gamma_{B _{nr_2}}= (\Sigma,\mathcal{A},\mathcal{A} ',R)$ where,\\
$\Sigma=\{a_2,b_1,c_1,d_1,g_0\}$,   $\mathcal{A} =\{A_1,A_2\}$ ,$\mathcal{A} ' =\{A_1\}$\\
{\small $A_1=
\begin{Bmatrix}
        \pstree[nodesep=1pt,levelsep=3ex]{\Tr{a}}
        {
            \Tr{b}
            \Tr{c}
            \Tr{d}
        }
        \end{Bmatrix}
        ,
        \hspace{2cm}    
        A_2=
     \begin{Bmatrix}       
        \pstree[nodesep=1pt,levelsep=6ex]{\Tr{b}}
        {}
        ,
        \pstree[nodesep=1pt,levelsep=6ex]{\Tr{c}}
        {}
        ,
        \pstree[nodesep=1pt,levelsep=6ex]{\Tr{d}}
        {}
        ,
        \pstree[nodesep=1pt,levelsep=6ex]{\Tr{g}}
        {}
\end{Bmatrix}$}
\\\\$R=\{B_1,B_2\}$, where\\\\
   $B_1=\{r_1,r_2,r_3\}$ and $B_2=\{r_4,r_5,r_6\}$\\
   $r_1=(\chi _1,C_1)$,   $r_2=(\chi _2,C_2)$,  $r_3=(\chi _3,C_3)$, $r_4=(\chi _1,C_4)$
   $r_5=(\chi _1,C_4)$,  $r_6=(\chi _1,C_4)$ ,where\\ 
   $\chi _1=(b,-,-)$,   $\chi _2=(c,-,-)$,  $\chi _3=(d,-,-)$\\   $C_1=(A_2,b,1)$,  $C_2=(A_2,c,1)$,  $C_3=(A_2,d,1)$,  $C_4=(A_2,g,1)$\\
\noindent \rule{\textwidth}{1pt}
\end{example}

